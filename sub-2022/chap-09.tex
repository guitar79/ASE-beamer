%%%%%%%%%% *** The Title %%%%%%%%%
\title[]{중위도 저기압\\\small{제9장}}

\begin{frame}[plain] %title page
	\titlepage
\end{frame}


\section{전선성 날씨}


\begin{frame}[t]{전선이란}
	\begin{tabular}{ll}
		\begin{minipage}[t]{0.45\textwidth}\scriptsize
			전선면: 서로 다른 기단 사이의 경계면\\
			전선 : 전선면과 지표면의 교선\\
			노르웨이 기상학자 들은 기단 간의 상호작용을 전투가 벌어지는 전선(battle line)에 비유하여, 전투 전선(battlefront) 처럼 전선(front)라고 이름붙였다.
		\end{minipage}	
		&
		\begin{minipage}[t]{0.5\textwidth} \scriptsize	
			전선에는 다섯 가지가 존재한다. \\
			온난 전선, 한랭 전선, 정체 전선, 폐색 전선, 건조 (전)선(Drylines)
			
		\end{minipage}
	\end{tabular}
\end{frame}



\begin{frame}[t]{온난 전선 (warm front)}
	\begin{tabular}{ll}
		\begin{minipage}[t]{0.45\textwidth}\scriptsize
			\begin{figure}[t]
				\includegraphics[trim=30 30 290 330, clip, page=267, width=0.9\textwidth]{\bookfile}
			\end{figure}
		\end{minipage}	
		&
		\begin{minipage}[t]{0.5\textwidth} \scriptsize	
			따뜻한 공기가 좀 더 차가운 공기로 이동할 때 생성\\
			이동 방향에 빨간 반원을 가진 붉은 색선으로 일기도에 표현\\
			상대적으로 완만한 경사를 가짐(약 1:200)\\
			
				\questionset{온난 전선이 접근하는 신호는 무엇이 있는가?}
			\solutionset{1) 권운 : 지면 전선 앞의 $1000 \rm{~km}$ 이상 떨어진 곳에서 형성\\
				2) 항공기의 비행운 : 맑은 날 비행운이 오랜 시간 지속될 때, 비교적 따뜻하고 습한 공기가 상승하고 있음을 알 수 있음. \newline}
			
			\questionset{당신이 전형적인 온난 전선의 지표 부분으로부터 $400 \rm{~km}$ 떨어진 지점에 서 있다면, 당신 머리 위 전선면의 높이는 얼마나 되는가?}
			\solutionset{온난 전선의 기울기는 약 1/200이므로 온난 전선의 높이는 약 $2 \rm{~km}$이다.\\
				반면 한랭 전선의 기울기는 약 1/100이므로 한랭 전선이었다면 높이는 약 $4 \rm{~km}$가 되었을 것이다.}
			
		\end{minipage}
	\end{tabular}
\end{frame}



\begin{frame}[t]{ 온난 전선 (warm front)}
	\begin{tabular}{ll}
		\begin{minipage}[t]{0.45\textwidth}\scriptsize
			\begin{figure}[t]
				\includegraphics[trim=50 350 240 50, clip, page=268, width=\textwidth]{\bookfile}
			\end{figure}
		\end{minipage}	
		&
		\begin{minipage}[t]{0.5\textwidth} \scriptsize	
			\questionset{온난 전선과 관련된 강수의 특징을 설명 하시오.}
			\solutionset{온난 전선이 접근하면 권운은 권층운으로, 권층운은 층운과 난층운으로 변화하하면서 구름 밑면의 높이가 점점 낮아다가, 비가 내리기 시작한다. 진행속도가 느리고 완만한 경사로 인해 온난 전선에서의 상승은 수평적으로 넓은 구조를 가지며, 이에 따라 장기간 넓은 지역에 적은 양의 강수를 내리게 하기 쉽다. \\
			그러나 위의 기단이 건조할 경우 구름 발달이 최소화되고 강수가 없을 수도 있다. 또한 보통 전선의 앞에서 비가 내린다.\\
			아주 습한 공기가 접근하는 온난 전선의 경우, 조건부 불안정한 공기가 충분히 상승하면 적란운과 뇌우를 만들기도 한다.  }
			
		\end{minipage}
	\end{tabular}
\end{frame}




\begin{frame}[t]{한랭 전선 (cold front)}
	\begin{tabular}{ll}
		\begin{minipage}[t]{0.45\textwidth}\scriptsize
			\begin{figure}[t]
				\includegraphics[trim=50 20 270 540, clip, page=269, width=\textwidth]{\bookfile}
			\end{figure}
		\end{minipage}	
		&
		\begin{minipage}[t]{0.5\textwidth} \scriptsize	
			따뜻한 공기로 차가운 공기가 진행할 때, 불연속이 나타나는 지점을 한랭 전선이라고 함.\\
			일기도 상에 진행방향으로 푸른색 삼각형에 푸른색 선으로 나타남.\\
			온난 전선의 2배의 경사를 가지고, 이동 속도가 더 빠름.\\
			
			\questionset{한랭 전선과 관련된 강수의 특징을 설명 하시오.}
			\solutionset{한랭 전선을 따라 따뜻하고 습한 공기의 강제적 상승이 빨라서 적란운이 발달하며, 많은 비와 격렬한 돌풍이 발생한다. 온난 전선보다 수평 거리가 짧기 때문에 강수의 강도는 강하며 시간은 짧다. }
		\end{minipage}
	\end{tabular}
\end{frame}




\begin{frame}[t]{뒷문 한랭 전선(backdoor cold front)}
	\begin{tabular}{ll}
		\begin{minipage}[t]{0.9\textwidth}\scriptsize
			\begin{figure}[t]
				\includegraphics[trim=350 260 40 310, clip, page=270, width=0.49\textwidth]{\bookfile}
				\includegraphics[trim=350 35 40 535, clip, page=270, width=0.49\textwidth]{\bookfile}
			\end{figure}
		\end{minipage}	
		&
		\begin{minipage}[t]{0.05\textwidth} \scriptsize	

		\end{minipage}
	\end{tabular}
		\scriptsize 
			미국의 봄에 북동부 지역에서 발생하여 서에서 동으로 이동하지 않고, 동에서 서로 이동하는 한랭 전선\\
			날씨가 차가워지고 습해짐

\end{frame}


\begin{frame}[t]{전선과 날씨}
	\begin{tabular}{ll}
		\begin{minipage}[t]{0.475\textwidth}\scriptsize
			\questionset{온난 전선과 관련된 전형적인 날씨를 설명하시오.}
			\solutionset{\begin{figure}[t]
				\includegraphics[trim=30 450 350 60, clip, page=269, width=0.9\textwidth]{\bookfile}
			\end{figure}}
		\end{minipage}	
		&
		\begin{minipage}[t]{0.475\textwidth} \scriptsize	
			\questionset{한랭 전선과 관련된 전형적인 날씨를 설명하시오.}
			\solutionset{\begin{figure}[t]
					\includegraphics[trim=50 50 340 550, clip, page=270, width=0.9\textwidth]{\bookfile}
			\end{figure}}
			
			
		\end{minipage}
	\end{tabular}
\end{frame}




\begin{frame}[t]{폐색 전선}
	\begin{tabular}{ll}
		\begin{minipage}[t]{0.475\textwidth}\scriptsize
			\begin{figure}[t]
				\includegraphics[trim=40 410 250 50, clip, page=271, width=\textwidth]{\bookfile}
			\end{figure}
			\questionset{폐색 전선은 어떻게 형성되는가?}
			\solutionset{일반적으로 한랭 전선의 이동 속도가 온난 전선보다 빠르므로 헌랭 전선이 온난 전선을 따라잡을 때 형성된다.}
		
		\end{minipage}	
		&
		\begin{minipage}[t]{0.475\textwidth} \scriptsize	
			\begin{figure}[t]
				\includegraphics[trim=40 260 250 375, clip, page=271, width=\textwidth]{\bookfile}
			\end{figure}
						
			\questionset{한랭형 폐색 전선과 온난형 폐색 전선을 비교하시오.}
			\solutionset{한랭형 폐색 전선은 온난 전선과 그 앞의 차가운 공기도 들어올린다. 초기에는 온난 전선에 의한 날씨와 유사하나 폐색이 발달하고 공기가 더 상승하면서 뇌우가 발생할 수 있다. 온난형 폐색전선은 전선 뒤의 공기가 앞선 차가운 공기보다 따뜻할 때 발생한다. 한랭형 폐색이 온난형 폐색보다 일반적이다. }
		\end{minipage}
	\end{tabular}
\end{frame}


\begin{frame}[t]{건조 전선(drylines)}
	\begin{tabular}{ll}
		\begin{minipage}[t]{0.45\textwidth}\scriptsize
			\begin{figure}[t]
				\includegraphics[trim=45 450 350 50, clip, page=273, width=\textwidth]{\bookfile}
			\end{figure}
		\end{minipage}	
		&
		\begin{minipage}[t]{0.5\textwidth} \scriptsize	
			전선은 다른 습도를 가진 공기로도 분리 가능 (건조 공기 밀도는 습한 공기 밀도보다 크다.\\
			온난건조한 공기가 온난습윤한 공기로 전진할 때 발달하는 전선\\
			건조 전선은 미국의 남부 대평원에서 나타남\\
			멕시코 지역의 cT 기단과 멕시코 만의 mT 기단이 만나서 형성되며, 뇌우가 많이 나타남 (스콜선과 연결됨)\\
			
			\questionset{건조 전선을 경계로 두 공기의 질을 비교하시오.}
			\solutionset{건조 전선 양쪽의 공기의 기온은 비슷하지만, 이슬점 온도는 건조 전선의 오른쪽 부분(mT)은 높고, 왼쪽 부분(cT)은 낮은 것을 볼 수 있다. }
			
		\end{minipage}
	\end{tabular}
\end{frame}



%\begin{frame}[t]{전선}
%	\begin{tabular}{ll}
%		\begin{minipage}[t]{0.475\textwidth}\scriptsize
%		\questionset{온난 전선을 정의하고, 온난 전선의 특징을 설명하시오.}
%		\solutionset{따뜻해서 가벼운 공기가 차가워서 무거운 공기를 밀 때 나타남
%			전선의 기울기가 1/200정도로 대단히 완만
%			전선이 접근하면 권운->권층운->고층운->층운 (300 km 거리)->난층운이 나타나고 비가 내림.
%			보통 늦은 진행속도와 매우 완만한 경사로 인해 수평적으로 넓은 형태의 상승이 나타나고, 넓은 지역에서 오랜 시간 동안 따뜻하고 적은 양의 강수가 전선 앞에서 내리게 됨.
%			하지만 상승하는 따뜻한 공기의 습도에 따라 구름이 거의 생기지 않을 수도 있고, 습도가 매우 높은 공기 덩이가 상승하게 되는 경우에는 적란운과 뇌우가 생성될 수 있음.
%			일기도 상에는 붉은색 반원으로 표현.}
%		\end{minipage}	
%		&
%		\begin{minipage}[t]{0.475\textwidth} \scriptsize	
%		
%			\questionset{한랭 전선을 정의하고 한랭 전선의 특징을 설명하시오.}
%			\solutionset{차가운 대륙성 극지방 공기가 따뜻한 공기로 덮인 지역으로 활발히 이동할 때 발생\\
%			지면 마찰은 전선의 지면에서의 속도를 상층에서의 전선 이동 속도보다 늦추지만, 
%			온난 전선에 비해 기울기가 2배 정도이고 이동 속도도 25 ~ 35 km/h 보다 빠른 35~ 50 km/h임. 
%			이는 온난 전선에 비해 한랭 전선의 날씨가 더 격렬하게 변함을 의미함.
%			한랭 전선의 도착 전에 생기는 구름은 고적운이며, 서쪽이나 북서쪽에서 발견. 
%			한랭 전선의 경우 따뜻하고 습한 공기의 강제적 상승이 너무 빨라서 방출된 잠열이 공기의 부력을 더욱 크게 하여 발달한 적란운이나 많은 강수나 격렬한 돌풍등이 전선에서 자주 나타나며 강수 시간은 짧음. 
%			일기도 상에서 푸른색 삼각점으로 나타냄.
%			한랭 전선이 지나간 후에는 보통 대륙성의 극기단에 의해 침강하는 공기가 지배하므로, 장기간 맑게 갠 밤이 이어지며 풍부한 복사 냉각이 발생하여 기온이 더욱 낮아짐.}
%			
%		\end{minipage}
%	\end{tabular}
%\end{frame}



%\begin{frame}[t]{전선}
%	\begin{tabular}{ll}
%		\begin{minipage}[t]{0.475\textwidth}\scriptsize
%		\questionset{정체 전선을 정의하고, 정체 전선의 특징을 설명하시오.}
%		\solutionset{전선의 서로 다른 측면의 기류가 차가운 기단 쪽으로나 따뜻한 기단 쪽으로 향하지 않고 거의 전선에 평행하게 나타나면 전선의 위치는 움직이지 않거나 아주 조금만 움직이게 되는데 이러한 경우를 정체 전선이라고 함. 정체 전선은 푸른색 삼각형과 붉은색 반원으로 표시됨. \newline}
%		
%		\questionset{정체 전선이 멈춰 있거나 천천히 움직이면서 어떻게 강수를 발생시키는가?}
%		\solutionset{정체 전선에서도 따뜻한 공기가 차가운 공기의 전선면 위로 상승하면서 약한 강수가 나타날 수 있음. 
%		특히 우리나라의 경우 장마철에 형성되는 정체 전선은 온도의 차이가 많이 나지만 둘다 해양성 기단이므로 강수가 많이 발생하는 편임. 
%		뿐만 아니라 정체 전선이 여러 날 동안 한 지역에 머무르게 되면 지속적인 강수로 인하여 홍수가 발생하는 경우도 있음.}
%
%		\end{minipage}	
%		&
%		\begin{minipage}[t]{0.475\textwidth} \scriptsize	
%			\questionset{폐색 전선을 정의하고, 폐색 전선의 특징을 설명하시오.}
%			\solutionset{한랭 전선의 속도는 온난 전선보다 빨라 폐색 전선이 형성되는데, 온도에 따라 두 가지 폐색 전선이 생성될 수 있음. 
%			한랭형 폐색 전선은 한랭 전선 뒤의 차가운 공기가 온난 전선 앞의 차가운 공기보다 더 차가운 경우 나타나며, 상층의 두 전선면 사이의 만나는 선이 하층의 전선보다 더 뒤쪽에 나타나 마치 한랭 전선과 비슷한 특성을 보임. 미국의 경우 로키 산맥의 동쪽에서 잘 나타남.
%			온난형 폐색전 선은 한랭 전선 뒤의 차가운 공기가 온난 전선 앞의 차가운 공기보다 더 따뜻한 경우 나타나며, 상층의 두 전선면 사이의 만나는 선이 하층의 전선보다 더 앞쪽에 나타남. 보통 태평양쪽에서 형성됨.
%			진행하는 방향 쪽을 향한 보라색 삼각형과 반원을 번갈아 선과 함께 그림. \newline}
%				
%			\questionset{건조 (전)선을 정의하고, 건조 (전)선의 특징을 설명하시오.}
%			\solutionset{수증기는 건조 공기보다 가볍기 때문에 건조한 공기는 상대적으로 무거워서 한랭 전선처럼 진행 방향에 있는 습한 공기를 강제적으로 상승시킴. 이러한 종류의 전선 경계가 지나가면 특별한 온도의 감소 없이도 습도는 급격히 떨어지게 됨. 
%			이러한 전선 경계와 관련된 형태를 건조 전선이라고 부름. 
%			미국에서는 봄과 여름에 대평원 서부에서 나타나는 데 북서부에서 발생한 건조한 대륙성 열대 기단(cT)이 멕시코 만의 습한 해양성 열대 기단(mT)를 만났을 때 발생함.}
%			
%			
%		\end{minipage}
%	\end{tabular}
%\end{frame}



\begin{frame}[t]{전선}
	\begin{tabular}{ll}
		\begin{minipage}[t]{0.45\textwidth}\scriptsize
		\questionset{온난 전선보다 한랭 전선에서의 날씨가 좋지 않은 이유는 무엇인가?}
		\solutionset{한랭 전선은 온난 전선에 비해 상승기류가 더 좁은 영역에서 집중적으로 나타나므로, 강수가 좁은 지역에서 짧은 시간 동안 집중적으로 나타남. \newline}
		
		\questionset{오랜 전조 뒤에 내린 비는 오래 지속되고, 짧은 전조 뒤에 내린 비는 빨리 그친다는 날씨 속담의 근거가 무엇인지 설명하라.}
		\solutionset{보통 온난 전선 앞에는 권운부터 시작하여 차츰 낮은 구름들이 나타나다 따뜻하고 약한 비가 오랫동안 내리는 반면, 한랭 전선 근처에서는 급격한 공기의 상승에 의한 적란운이 많이 생겨 차고 강한 비가 짧게 내리기 때문임.}
		\end{minipage}	
		&
		\begin{minipage}[t]{0.5\textwidth} \scriptsize	
			
		\end{minipage}
	\end{tabular}
\end{frame}




 

\section{중위도저기압과 극전선 이론}




\begin{frame}[t]{중위도 저기압}
	\begin{tabular}{ll}
		\begin{minipage}[t]{0.45\textwidth}\scriptsize
			\begin{figure}[t]
				\includegraphics[trim=250 40 40 420, clip, page=276, width=\textwidth]{\bookfile}
			\end{figure}
		\end{minipage}	
		&
		\begin{minipage}[t]{0.5\textwidth} \scriptsize	
			중위도 저기압 : 지름이 $1000 \rm{~km}$가 넘는 저기압
			보통 서쪽에서 동쪽으로 이동 \\
			저기압 중심에 앞선 온난 전선과 뒤따르는 한랭 전선을 가짐
			강수가 잦음\\
			
			\questionset{초기 노르웨이 기상학자들의 극전선(polar front) 이론을 설명하시오.}
			\solutionset{서로 다른 성질을 가진 두 기단은 속력도 방향도 조금씩 차이가 나게 되며 결국 두 기단이 서로 충돌하게 되는데, 기단들이 상호작용하는 면을 전선면, 전선면과 지표면이 만나는 지점을 전선이라고 함.
				극전선이란 따뜻한 아열대 공기와 극지방의 차가운 공기가 만나는 전선면이 지표면과 만나는 지점으로 지표면에서는 불연속적으로 나타나지만 상층에서는 거의 연속적인 전선면으로 나타남.
				보통 중위도 저기압 또는 온대 저기압이 만들어지는 과정을 기술하는 이론 또는 모형을 극전선 이론 또는 노르웨이식 저기압 모형이라고 한다. \newline}
			
		\end{minipage}
	\end{tabular}
\end{frame}




\begin{frame}[t]{중위도 저기압의 생애}
	\begin{tabular}{ll}
		\begin{minipage}[t]{0.475\textwidth}\scriptsize
			\begin{figure}[t]
				\includegraphics[trim=245 70 30 250, clip, page=274, width=0.9\textwidth]{\bookfile}
			\end{figure}
		\end{minipage}	
		&
		\begin{minipage}[t]{0.475\textwidth} \scriptsize	
			A. 형성: 서로 다른 온도의 기단이 극전선에 평행하게 반대의 방향으로 움직이면서 형성\\
			B. 파동 발달 : 적절한 상황에서 경계부가 파 모양을 띠고, 폭풍이 강해짐에 따라 파의 모양이 변함 \\
			C. 저기압성 순환 발달 : 따뜻한 공기는 극방향으로 이동하면서 온난 전선을 형성 하고, 차가운 공기는 적도방향으로 이동하면서 한랭 전선 형성되며, 저기압은 파의 봉우리 부근에 위치\\
			D. 성숙 단계 : 주변 기압이 약간 하강함, 전선의 모양이 변하고, 전선 면에서 나타나는 일기가 나타남\\
			E. 폐색: 찬 공기가 더 빨리 이동히야 차가운 공기가 따뜻한 공기를 따라 잡음\\
			F. 저기압 소멸
			
		\end{minipage}
	\end{tabular}
\end{frame}




%\begin{frame}[t]{극전선 이론}
%	\begin{tabular}{ll}
%		\begin{minipage}[t]{0.45\textwidth}\scriptsize
%			\begin{figure}[t]
%				\includegraphics[trim=50 290 50 50, clip, page=218, width=\textwidth]{\bookfile}
%			\end{figure}
%		\end{minipage}	
%		&
%		\begin{minipage}[t]{0.5\textwidth} \scriptsize	

			
			
%			\questionset{중위도 저기압 형성의 단계에 대해 설명하시오.}
%			\solutionset{전형적인 중위도 저기압의 발달 주기는 전선의 발달, 파동의 발달, 저기압성 흐름의 발달, 저기압의 형성, 폐색 전선의 발달, 온대 저기압의 소멸로 나타난다. 각 단계별 특징은 아래와 같다.
%			1) 전선의 발달
%			밀도가 다른 두 기단들이 거의 전선에 평행하게 서로 반대 방향으로 움직임
%			- 북동풍의 보통 대륙성 극지방 공기
%			- 남서풍의 해양성 열대기단
%			이 두 기단에 의한 전선면은 길이가 수 백 km에 달하는 긴 파동의 형태를 이룸
%			2) 파동의 발달
%			불연속 전선면이 파동 모양으로 뒤틀리기 시작함. 
%			동서류(zonal flow)의 뒤틀림 원인: 불균형한 지형(산), 온도 차이(바다와 육지 사이), 해류의 영향
%			지상 저기압의 발달은 상층 기류가 북쪽에서 남쪽으로 넓게 흐를 때, 즉 골과 능이 번갈아 발생하여 큰 규모의 파를 만들 때 잘 일어난다.
%			
%			3) 저기압성 흐름의 발달과 저기압의 형성
%			파의 발달에 따라 따뜻한 공기는 차가운 공기 위로 상승하고, 차가운 공기는 적도쪽으로 파고듦. 이러한 반시계 방향의 저기압성 흐름은 파가 저기압으로 발달하도록 도움. 
%			이러한 저기압성 순환의 발달은 저기압의 중심에서 상승기류가 발달하게 함.
%			따뜻한 공기가 상승하는 지역에서는 온난 전선이, 차가운 공기가 침투하는 지역에서는 한랭 전선이 생성됨.
%			결국 전선의 오른쪽에서는 온난 전선이 생성되고, 전선의 왼쪽에서는 한랭 전선이 생성됨.
%			4) 폐색 전선의 발달 
%			한랭 전선은 온난 전선보다 빠르게 이동하므로, 온난 전선에 가까워지면서 따뜻한 공기를 상층으로 들어올리기
%			시작하는 폐색 과정이 일어나고, 이 과정에서 폐색 전선이 형성됨.
%			폐색이 시작되면 폭풍은 강해지고, 중심기압은 떨어지고, 풍속은 증가함.
%			5) 온대 저기압의 소멸
%			경사진 전선이 강제적으로 상승하게 되면 기압 경도는 약해지고, 대립하던 두 기단 사이에 존재하는 수평 온도
%			차이는 없어지게 되고 이때 저기압은 에너지 원천이 다 사라지게 됨.
%			마찰은 지면 흐름을 느리게 하고 반시계 방향의 흐름은 서서히 사라지게 됨.}
			
			
%		\end{minipage}
%	\end{tabular}
%\end{frame}






\section{중위도 저기압의 이상적인 날씨}



\begin{frame}[t]{중위도 저기압}
	\begin{tabular}{ll}
		\begin{minipage}[t]{0.45\textwidth}\scriptsize
			\begin{figure}[t]
				\includegraphics[trim=50 50 230 290, clip, page=278, width=\textwidth]{\bookfile}
			\end{figure}
		\end{minipage}	
		&
		\begin{minipage}[t]{0.5\textwidth} \scriptsize	
			\questionset{순전하는 바람과 역전하는 바람에 대해 구분하여 설명하시오.}
			\solutionset{전선의 통과에 따른 풍향의 전환에는 순전과 반전의 용어를 사용함.\\
				순전은 풍향이 시계 방향으로 바뀌는 것을 말하는데 온난 전선과 한랭 전선이 차례로 통과하면서 나타남. \\
				반전은 풍향이 반시계 방향으로 전환하는 것을 말하며 저기압의 북쪽에 위치한 지역에서 나타남.}
				
			\end{minipage}
		\end{tabular}
	\end{frame}
	
	


\section{상층기류와 저기압 생성}	
	
	\begin{frame}[t]{상층의 발산, 수렴}
		\begin{tabular}{ll}
			\begin{minipage}[t]{0.5\textwidth}\scriptsize
				\begin{figure}[t]
					\includegraphics[trim=50 480 210 50, clip, page=280, width=\textwidth]{\bookfile}
				\end{figure}
			\end{minipage}	
			&
			\begin{minipage}[t]{0.45\textwidth} \scriptsize	
				\questionset{속도 발산과 속도 수렴이란 무엇인가?}
				\solutionset{자동차 요금소 앞에서 자동차는 속력을 줄이게 되고 일정 구간에 있는 자동차들은 수가 늘어난다. 하지만 요금소를 지나게 되면 자동차는 속력이 증가하게 되고 일정 구간에 있는 자동차의 수는 줄어들게 된다. \\
					이와 같이 공기가 풍속이 빠른 지역에 들어가면 속력이 증가하여 바깥으로 빠져나가게(발산) 되고, 반대로 공기가 풍속이 느린 지역에 들어가면 속력이 감소하여 공기가 쌓이게(수렴) 됨. \\
					상층 대기에서 일어나는 발산은 속도 발산 외에도 공기 흐름이 수평적으로 퍼지는 방향성 발산과 공기의 질량에 의해 나타나는 회전량인 와도가 영향을 주지만 주로 속도 발산에 의해 일어남. }
					
				\end{minipage}
			\end{tabular}
		\end{frame}
		
		
		
	
	\begin{frame}[t]{제트 기류와 지상 저기압}
		\begin{tabular}{ll}
			\begin{minipage}[t]{0.55\textwidth}\scriptsize
				\begin{figure}[t]
					\includegraphics[trim=220 400 50 50, clip, page=283, width=\textwidth]{\bookfile}
				\end{figure}
			\end{minipage}	
			&
			\begin{minipage}[t]{0.4\textwidth} \scriptsize	
				\questionset{상층 일기도가 주어졌을 경우, 예보자들이 저기압 발생 지점을 찾기 위해 보는 곳은 어디이고, 고기압은 보통 상층일기도의 어느 부분에서 생성되는가?}
				\solutionset{중위도에서 지상 저기압은 일반적으로 상층 발산이 일어나는 상층의 골 동쪽에서 형성되고, 지상 고기압은 보통 상층 수렴이 일어나는 상층의 마루 동쪽에서 일어남. \\
					상층 발산이 지상에서의 수렴보다 크다면 지상 기압은 떨어지고, 저기압성 폭풍은 강해짐.}
					
				\end{minipage}
			\end{tabular}
		\end{frame}
		
			
			
			



\section{중위도 저기압은 어디에서 생성되는가?}



\begin{frame}[t]{저기압의 생성과 이동}
	\begin{tabular}{ll}
		\begin{minipage}[t]{0.475\textwidth}\scriptsize
			\begin{figure}[t]
				\includegraphics[trim=50 515 320 50, clip, page=284, width=\textwidth]{\bookfile}
			\end{figure}
		\end{minipage}	
		&
		\begin{minipage}[t]{0.475\textwidth} \scriptsize	
			\begin{figure}[t]
				\includegraphics[trim=350 50 50 520, clip, page=284, width=\textwidth]{\bookfile}
			\end{figure}
					
		\end{minipage}
	\end{tabular}
		태평양 연안, 로키산맥 동쪽, 맥시코 만, 대서양에서 생성\\
		저기압성 폭풍으로 앨버타 클리퍼, 팬핸들 훅, 노이스터 등이 있다.

\end{frame}






\section{현대의 관점: 컨베이어 밸트 모형}


\begin{frame}[t]{컨베이어 벨트 모형}
	\begin{tabular}{ll}
		\begin{minipage}[t]{0.55\textwidth}\scriptsize
			\begin{figure}[t]
				\includegraphics[trim=50 400 70 50, clip, page=286, width=\textwidth]{\bookfile}
			\end{figure}
		\end{minipage}	
		&
		\begin{minipage}[t]{0.4\textwidth} \scriptsize	
			\questionset{현대적 관점의 컨베이어 벨트 모형이란 무엇인가?}
			\solutionset{1) 온난 컨베이어 벨트
			멕시코 만에서 중위도 저기압의 온난 구역으로 따뜻하고 습한 공기를 수송.
			북쪽으로 흘러가면서 수렴에 의해 천천히 기류가 상승함. 온난 전선의 경사진 경계에 도달한 기류는 전선면의 아래에 놓인 차가운 공기 위로 빠르게 상승하면서 단열 팽창에 의해 넓은 구름 밴드와 강수를 만듬. 대류권 중층에 도착한 이 기류는 동쪽으로 돌아서 상층의 일반적인 서풍과 결합함. 온난 컨베이어 벨트는 중위도 저기압에서 강수를 일으키는 주된 공기 흐름임.\\
}
			
		\end{minipage}
	\end{tabular}
\end{frame}



\begin{frame}[t]{컨베이어 벨트 모형}
	\begin{tabular}{ll}
		\begin{minipage}[t]{0.55\textwidth}\scriptsize
			\begin{figure}[t]
				\includegraphics[trim=50 400 70 50, clip, page=286, width=\textwidth]{\bookfile}
			\end{figure}
		\end{minipage}	
		&
		\begin{minipage}[t]{0.4\textwidth} \scriptsize	
				{2) 한랭 컨베이어 벨트 
				온난 전선의 앞쪽 표면에서 시작하는 기류. 저기압의 중심을 향해 서쪽으로 붐. 온난 컨베이어 벨트 아래에 흐르는 이 공기는 강수 발생으로 인한 증발에 의해 습해짐. 기류가 저기압의 중심에 접근함에 따라 수렴 운동은 이 기류를 상승시키고, 공기는 포화되고 저기압성 강수가 발생함. 대류권에 도착하면 일부는 저기압 주변에 저기압 형태로 회전하며 성숙한 폭풍계를 나타내는 콤마 머리 모향을 만들고, 나머지 기류들은 오른쪽으로 돌아서 일반적인 서풍 기류가 됨. 이는 온난 컨베이어 벨트의 흐름과 평행하게 되고 강수를 발생시킴. }
			
		\end{minipage}
	\end{tabular}
\end{frame}





\begin{frame}[t]{컨베이어 벨트 모형}
	\begin{tabular}{ll}
	\begin{minipage}[t]{0.55\textwidth}\scriptsize
		\begin{figure}[t]
			\includegraphics[trim=50 400 70 50, clip, page=286, width=\textwidth]{\bookfile}
		\end{figure}
	\end{minipage}	
	&
	\begin{minipage}[t]{0.4\textwidth} \scriptsize		
		{3) 건조 컨베이어 벨트
			건조한 기류는 대류권 최상층에서 발생함. 상대적으로 차갑고 건조하며, 이 기류는 저기압에 들어가면서 두 부분으로 나누어짐. 기류의 한 부분은 한랭 전선면 아래로 하강하고, 그 결과 한랭 전선이 통과하면서 일반적인 맑고 차가운 날씨가 이어짐. 뿐만 아니라 이 기류는 한랭 전선을 경계로 나타나는 강한 온도 차이를 유지시켜줌. 건조 컨베이어 벨트의 또 다른 기류는 서풍을 유지하고 드라이 슬롯(맑게 갠 지역)을 형성함. 이 드라이 슬롯은 콤마 구름 형태의 머리와 꼬리를 분리함.}
		\end{minipage}
	\end{tabular}
\end{frame}




\begin{frame}[t]{위성 영상에 나타난 콤마 꼬리}
	\begin{tabular}{ll}
		\begin{minipage}[t]{0.55\textwidth}\scriptsize
			\begin{figure}[t]
				\includegraphics[trim=40 490 320 50, clip, page=279, width=\textwidth]{\bookfile}
			\end{figure}
		\end{minipage}	
		&
		\begin{minipage}[t]{0.4\textwidth} \scriptsize	
			
			
		\end{minipage}
	\end{tabular}
\end{frame}





\section{고기압성 날씨와 저지 고기압}



\begin{frame}[t]{저지 고기압}
	\begin{tabular}{ll}
		\begin{minipage}[t]{0.45\textwidth}\scriptsize
			\begin{figure}[t]
				\includegraphics[trim=320 0 50 450, clip, page=288, width=\textwidth]{\bookfile}
			\end{figure}
		\end{minipage}	
		&
		\begin{minipage}[t]{0.5\textwidth} \scriptsize	
			저지 고기압(blocking high): 상층 저기압의 동쪽 방향 이동을 막는 상층의 정체된 고기압을 저지 고기압이라 함.\\
			대기 오염 측면에서는 저지 고기압이 느리게 진행하면서, 고기압 중심에서의 강하로 기온 역전이 일어나고 오염물질이 빠져 나가지 못하도록 하고, 상대적으로 약한 바람이 불어 오염된 공기가 덜 확산되어 대기오염이 증가함. \\
			
			\questionset{저지 고기압이 날씨에 영향을 미치는 2가지 방법은 무엇인가?}
			\solutionset{1) 큰 저지 고기압은 겨울철 차가운 공기로 인해 한파를 불러올 수 있음.\\
			2) 동서 방향의 흐름(zonal flow)을 가로막아 공기의 흐름을 남쪽이나 북쪽으로 편향시켜 보내면서 저기압의 이동을 막아서 한 지역은 가뭄이, 다른 지역은 홍수가 나게 됨.}
			
		\end{minipage}
	\end{tabular}
\end{frame}




\begin{frame}[t]{분리 저기압}
	\begin{tabular}{ll}
		\begin{minipage}[t]{0.45\textwidth}\scriptsize
			\begin{figure}[t]
				\includegraphics[trim=40 490 350 50, clip, page=289, width=\textwidth]{\bookfile}
			\end{figure}
		\end{minipage}	
		&
		\begin{minipage}[t]{0.5\textwidth} \scriptsize	
			\questionset{분리 저기압(cut-off lows)이 날씨에 영향을 미치는 방법은 무엇인가?}
			\solutionset{서에서 동으로 부는 제트기류의 흐름에서 일부분이 분리되어 나와 생성되며, 고기압 시스템과 마찬가지로 블로킹 패턴을 생성.
			분리되어 나온 흐름은 상층 공기의 흐름과 연결 없이 몇 일간 그 지점에 머무르게 됨. 이에 따라 우중충한 날씨가 이어져 많은 강수를 내리게 하기도 함}
			
		\end{minipage}
	\end{tabular}
\end{frame}





\section{중위도 저기압 사례 연구}


\begin{frame}[t]{사례 연구}
	\begin{tabular}{ll}
		\begin{minipage}[t]{0.475\textwidth}\scriptsize
			\begin{figure}[t]
				\includegraphics[trim=40 30 240 360, clip, page=290, width=\textwidth]{\bookfile}
			\end{figure}
		\end{minipage}	
		&
		\begin{minipage}[t]{0.475\textwidth}\scriptsize
	\begin{figure}[t]
		\includegraphics[trim=260 290 40 50, clip, page=290, width=\textwidth]{\bookfile}
	\end{figure}
\end{minipage}	

	\end{tabular}
\end{frame}


\begin{frame}[t]{사례 연구}
	\begin{tabular}{ll}
		\begin{minipage}[t]{0.475\textwidth}\scriptsize
			\begin{figure}[t]
				\includegraphics[trim=40 340 240 50, clip, page=291, width=\textwidth]{\bookfile}
			\end{figure}
		\end{minipage}	
		&
		\begin{minipage}[t]{0.475\textwidth} \scriptsize	
			
			
		\end{minipage}
	\end{tabular}
\end{frame}
