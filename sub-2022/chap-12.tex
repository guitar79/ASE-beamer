%%%%%%%%%% *** The Title %%%%%%%%%%
\title[]{기상분석과 예보\\\small{제12장}}

\begin{frame}[plain] %title page
	\titlepage
\end{frame}


\section{기상산업: 간략한 개관}


\begin{frame}[t]{일기 예보}
	\begin{tabular}{ll}
		\begin{minipage}[t]{0.55\textwidth}\scriptsize
			\begin{figure}[t]
				\includegraphics[trim=0 470 260 0, clip, 
				page=356, width=0.9\textwidth]{\bookfile}
			\end{figure}
		\end{minipage}	
		&
		\begin{minipage}[t]{0.4\textwidth} \scriptsize	
			\begin{figure}[t]
				\includegraphics[trim=50 400 375 60, clip, 
				page=357, width=0.9\textwidth]{\bookfile}
			\end{figure}

		\end{minipage}
	\end{tabular}
\end{frame}



\begin{frame}[t]{일기 예보}
	\begin{tabular}{ll}
		\begin{minipage}[t]{0.45\textwidth}\scriptsize
			\begin{figure}[t]
				\includegraphics[trim=30 50 350 410, clip, 
				page=355, width=\textwidth]{\bookfile}
			\end{figure}
		\end{minipage}	
		&
		\begin{minipage}[t]{0.5\textwidth} \scriptsize	
			\questionset{일기예보(weather forecast)란 무엇인가?}
			\solutionset{미래의 기상 상태(온도, 습도, 풍향, 풍속, 맑고 흐림 등과 같은 기상 변수)에 대한 과학적 추정}
			
			\questionset{우리나라의 기상청과 같은 역할을 하는 미국 국립기상청(NWS)의 임무는?}
			\solutionset{NWS는 NOAA(National Oceanic and Atmospheric Administration)의 하부 조직으로 기상에 관한 자료의 수집과 배포와 관련된 일을 맡고 있음. 특히 뇌우, 홍수, 태풍, 토네이도, 겨울철 기상예보, 이상 고온 등의 재해 기상에 대한 예보가 가장 중요한 임무임.}
			
			\questionset{일기 예보 및 기상재해 경보를 위한 3단계 과정을 설명하시오.}
			\solutionset{1) 세계 각지로부터 기상 자료를 수집하여 전송, 전 지구 규모의 자료 생성(기상 분석)
					2) 다양한 기술을 이용하여 대기의 미래 상태를 예측(일기 예보)
					3) 생산된 예보자료를 대중에게 전파}

		\end{minipage}
	\end{tabular}
\end{frame}

\section{기상 자료의 수집}

\begin{frame}[t]{기상 자료 수집}
	\begin{tabular}{ll}
		\begin{minipage}[t]{0.6\textwidth}\scriptsize
			\begin{itemize}
				\item WMO(World Meteorological Organization)에서 세계의 기상 자료를 수집하여 제공.			
				
			\end{itemize}
		\end{minipage}	
		&
		\begin{minipage}[t]{0.35\textwidth} \scriptsize	
			\questionset{기상 자료 수집 기술의 제한점을 설명하시오.}
			\solutionset{기상 자료 수집 기술의 제한: 
				1) 관측기기의 불완전성
				2) 자료 전송 오차로 관측된 값이 실제값과 차이를 보임
				3) 해양/산악 지역의 관측 자료의 부족(관측이 시공간적으로 고르게 이루어지지 않음)}

			\questionset{기상 분석이란 무엇인가?}
			\solutionset{일기 예보를 위해 현재의 기상 상태를 정확하게 알아내는 것으로 자료 수집, 전송, 분류를 포함. 결과물로 종관 일기도가 만들어짐.}

		\end{minipage}
	\end{tabular}
\end{frame}



\begin{frame}[t]{지상 관측}
	\begin{itemize}\scriptsize
		\item 육상 관측소(만개 이상), 선박(7천 척), 해상 부이(수백 개), 유조선에서 하루 네 차례 관측 
		\item 육상 자동 기상 관측 시스템(ASOS: Automated Surface Observing System)을 이용하여 온도, 이슬점, 바람, 시정거리, 강수량, 맑고 흐림 상태 탐지
	\end{itemize}
	\begin{tabular}{ll}
		\begin{minipage}[t]{0.65\textwidth}\scriptsize
			\begin{figure}[t]
				\includegraphics[trim=50 50 210 520, clip, 
				page=358, width=\textwidth]{\bookfile}
			\end{figure}
		\end{minipage}	
		&
		\begin{minipage}[t]{0.3\textwidth} \scriptsize	
			\begin{figure}[t]
				\includegraphics[trim=50 350 320 50, clip, 
				page=358, width=\textwidth]{\bookfile}
			\end{figure}

		\end{minipage}
	\end{tabular}
\end{frame}



\begin{frame}[t]{상층 관측}
	\begin{tabular}{ll}
		\begin{minipage}[t]{0.4\textwidth}\scriptsize
			\begin{figure}[t]
				\includegraphics[trim=250 410 50 50, clip, 
				page=349, width=\textwidth]{\bookfile}
			\end{figure}
		\end{minipage}	
		&
		\begin{minipage}[t]{0.55\textwidth} \scriptsize	
			\begin{itemize}
				\item 라디오 존데 
				: 가벼운 풍선 형태의 관측 기기로 온도, 습도, 기압 등을
				  측정
				: 전 세계 약 1300여개 활용, 매일 두 차례 관측      
				: 90분 상승 후, 35km 상공에 도착하여 터짐
		  
				\item 레윈 존데
				: 라디오존데와 같은 원리, 풍향과 풍속을 측정
		  
				\item 비행기 
				: 비행하는 동안 바람과 기온 측정
		  
				\item 위성
				: 가시광과 적외선 탐지기를 장착하여 기온과 습도 자료
				  수집, 구름과 수증기의 이동 경로 추적

				\item 윈드 프로파일러(wind profiler)
				  : 지상으로부터 약 10 km 고도까지 풍속과
				   풍향을 측정하는 레이더
				  : 6분마다 한 번씩 측정
			 
				\item 도플러 레이더(Doppler radar)
				  : 바람과 강수량 측정하므로 악기상 추적에
				   사용
				  : 강수의 패턴과 강수지역의 이동 경로 탐지
			 
					  
			\end{itemize}

		\end{minipage}
	\end{tabular}
\end{frame}




\section{일기도: 대기 상태의 예측}



\begin{frame}[t]{종관 일기도(Synoptic weather map)}
	\begin{itemize} \scriptsize	
		\item 특정 시점의 온도, 습도, 기압, 풍향, 풍속 등의 대기 상태를 종합하여 자동화 시스템을 통해 작성한 후 날씨 분석가의 오류 수정과 첨삭을 거쳐 제시(하루 여러 차례 가능)
		\item 850 hPa, 700 hPa, 500 hPa, 300 hPa, 200 hPa 고도에 대해서는 하루 2번 일기도 생산
	\end{itemize}

	\begin{tabular}{ll}
		\begin{minipage}[t]{0.475\textwidth}\scriptsize
			\begin{figure}[t]
				\includegraphics[trim=50 450 225 50, clip, 
				page=360, width=\textwidth]{\bookfile}
			\end{figure}
		\end{minipage}	
		&
		\begin{minipage}[t]{0.475\textwidth} \scriptsize	
			\begin{figure}[t]
				\includegraphics[trim=50 50 300 360, clip, 
				page=360, width=\textwidth]{\bookfile}
			\end{figure}

		\end{minipage}
	\end{tabular}
\end{frame}



\begin{frame}[t]{기상 분석}
	\begin{tabular}{ll}
		\begin{minipage}[t]{0.3\textwidth}\scriptsize
			\begin{figure}[t]
				\includegraphics[trim=250 410 50 50, clip, 
				page=349, width=\textwidth]{\bookfile}
			\end{figure}
		\end{minipage}	
		&
		\begin{minipage}[t]{0.65\textwidth} \scriptsize	
			\questionset{지상 일기도의 주된 용도는 무엇인가?}
			\solutionset{1) 고,저기압이나 전선 등의 위치를 파악하고 바람, 기온, 기상현상 등의 분포를 파악한다.
						2) 국지적인 기상현상을 추적하는 데 활용한다.
						3) 현재의 날씨 파악을 통해 단기간 기상변화를 예측하는 데 이용한다.}

			\questionset{지상 일기도에서 알 수 있는 정보에는 무엇이 있는가?}
			\solutionset{지상일기도에는 온도, 이슬점, 기압, 기압변화 경향, 운량(고도, 형태, 양), 풍향과 풍속, 
						현재 및 과거의 기상 상태 등의 정보와 등압선(1000 hPa을 기준으로 4 hPa 간격)과 
						전선 정보가 제공된다}



			\questionset{지상 일기도가 상층 일기도에 비해 가지는 장점은 무엇인가?}
			\solutionset{1) 상층에 비해 관측 공간 및 시간이 조밀하여 자료의 양이 많고, 정확도가 높다.
						2) 하루에 여러 차례 일기도를 작성할 수 있음.
						3) 우리가 살고 있는 지상의 일기 상태를 알려줌
						4) 상층일기도에는 없는 전선을 그릴 수 있음}
		\end{minipage}
	\end{tabular}
\end{frame}



\begin{frame}[t]{일기 기호}
	\begin{tabular}{ll}
		\begin{minipage}[t]{0.6\textwidth}\scriptsize
			\begin{figure}[t]
				\includegraphics[trim=35 485 300 47, clip, 
				page=361, width=\textwidth]{\bookfile}
			\end{figure}
		\end{minipage}	
		&
		\begin{minipage}[t]{0.35\textwidth} \scriptsize	
			\begin{itemize}
				\item 				
				\item 
				\item 
				
			\end{itemize}

		\end{minipage}
	\end{tabular}
\end{frame}



\begin{frame}[t]{등압선 그리기}
	\begin{tabular}{ll}
		\begin{minipage}[t]{0.2\textwidth}\scriptsize
			\begin{figure}[t]
				\includegraphics[trim=30 35 60 320, clip, 
				page=361, width=\textwidth]{\bookfile}
			\end{figure}
		\end{minipage}	
		&
		\begin{minipage}[t]{0.75\textwidth} \scriptsize	
			\begin{itemize}
				\item 등압선은 1000 hPa을 기준으로 4 hPa 간격으로 그린다.
				\item 그리기 쉬운 곳(자료가 많은 곳)부터 그려 나간다.
				\item 등압선은 반드시 폐곡선이거나 일기도의 가장자리에서 끝나도록 그린다.
				\item 등압선은 서로 교차하지 않게 그린다.
				\item 등압선은 갈라지거나 합쳐지지 않게 그린다.	
				\item 관측값이 없는 경우는 내삽, 외삽법의 원리로 두 지점의 간격을 고려하여 부드러운 곡선으로 그린다.
				\item 등압선은 부드러운 곡선으로 그린다. 
				\item 전선이 있는 경우 전선을 경계로 등압선이 꺾이도록 그린다.
				\item 기압의 표시는 한 선으로 연결된 등압선은 양쪽 끝에, 폐곡선인 경우는 위쪽 중앙에 등압선을 끊고 기입한다.
				\item 고기압의 중심은 청색으로 고(H:high pressure), 저기압은 적색으로 저(L: low pressure) 자를 표시한다.  
			\end{itemize}

		\end{minipage}
	\end{tabular}
\end{frame}


\begin{frame}[t]{간략화한 일기도}
	\begin{tabular}{ll}
		\begin{minipage}[t]{0.7\textwidth}\scriptsize
			\begin{figure}[t]
				\includegraphics[trim=30 35 60 320, clip, 
				page=361, width=\textwidth]{\bookfile}
			\end{figure}
		\end{minipage}	
		&
		\begin{minipage}[t]{0.25\textwidth} \scriptsize	
			\questionset{지상 일기도에서 전선의 위치를 파악하는 방법을 설명하시오.}
			\solutionset{1) 온도의 변화가 급격하게 일어나는 곳
			2) 근 거리에서 풍향이 시계 뱡향으로 90° 정도 크게 바뀌는 곳
			3) 이슬점(습도)이 크게 변하는 곳
			4) 구름과 강수 패턴을 활용}
		\end{minipage}
	\end{tabular}
\end{frame}


\begin{frame}[t]{상층 일기도}
	\begin{tabular}{ll}
		\begin{minipage}[t]{0.3\textwidth}\scriptsize
			\begin{figure}[t]
				\includegraphics[trim=250 410 50 50, clip, 
				page=349, width=\textwidth]{\bookfile}
			\end{figure}
		\end{minipage}	
		&
		\begin{minipage}[t]{0.65\textwidth} \scriptsize	
			\begin{itemize}
				\item 상층에서는 특정 고도에서 기압을 나타내는 대신, 특정 기압을 갖는 등고선으로 기압 분포를 나타냄. 
				\item 지형의 높고 낮음을 표시한 것과 유사하여 고도가 높은 곳은 언덕, 낮은 곳은 계곡에 해당함.
				\item 즉, 특정 기압을 갖는 고도가 높은 곳은 고기압에 해당되며, 고도가 낮은 곳은 저기압에 해당.
				%\item 500 hPa을 갖는 고도를 측정하여 일기도에 등고선으로 표시함.
				\item 동일한 고도에서 비교하면, 고도 값이 높은 지역이 낮은 지역보다 기압이 크다. 그러므로 고기압에 해당함.
					
			\end{itemize}

		\end{minipage}
	\end{tabular}
\end{frame}



\begin{frame}[t]{850 hPa 일기도}
	\begin{tabular}{ll}
		\begin{minipage}[t]{0.5\textwidth}\scriptsize
			\begin{figure}[t]
				\includegraphics[trim=50 470 300 50, clip, 
				page=362, width=\textwidth]{\bookfile}
			\end{figure}
		\end{minipage}	
		&
		\begin{minipage}[t]{0.45\textwidth} \scriptsize	
			\begin{itemize}\scriptsize
				\item 850 hPa면은 약 1.5 km 상공의 등압면에 해당하며, 대류권 하부에 해당함.
				\item 이 고도는 마찰이 크고 난류가 활발한 행성 경계층의 상단에 해당.
				\item 고도 30 m 간격(검은선), 기온 3 ℃ 간격(붉은선)으로 나타냄				
			\end{itemize}
		\end{minipage}
	\end{tabular}
\end{frame}



\begin{frame}[t]{850 hPa 일기도}
	\begin{tabular}{ll}
		\begin{minipage}[t]{0.475\textwidth}\scriptsize
			\begin{figure}[t]
				\includegraphics[trim=395 550 50 55, clip, 
				page=362, width=\textwidth]{\bookfile}
			\end{figure}
			
		\end{minipage}	
		&
		\begin{minipage}[t]{0.475\textwidth} \scriptsize	
			\begin{itemize}\scriptsize	
				\item 상대습도를 표시하는 경우, 습도가 70 \% 이상인 지역을 초록색으로 표시		

			\end{itemize}

		\end{minipage}
	\end{tabular}
\end{frame}


\begin{frame}[t]{850 hPa 일기도}
	\begin{tabular}{ll}
		\begin{minipage}[t]{0.475\textwidth}\scriptsize
			\begin{figure}[t]
				\includegraphics[trim=50 285 370 315, clip, 
				page=362, width=\textwidth]{\bookfile}
			\end{figure}
		\end{minipage}	
		&
		\begin{minipage}[t]{0.475\textwidth} \scriptsize	
			

			\begin{itemize}\scriptsize	
				\item 한랭 이류와 온난 이류가 발생하는 지역을 찾기 위해 850 hPa 일기도를 활용.
				\item 하층의 한랭이류는 상층의 대기를 하강하게 하므로, 대기 안정도를 증가시키며, 맑은 날씨를 가져오게 됨.
				\item 하층의 온난이류는 하층의 대기를 상승시키게 하므로, 구름이 형성되고 강수를 유발함.
				\item 즉, 기온을 예보하는 데 활용할 수 있음. 온난이류가 있는 지역은 기온이 높을 것이며, 한랭이류가 있는 지역은 기온이 낮을 것으로 예보할 수 있음. 
				
			\end{itemize}

		\end{minipage}
	\end{tabular}
\end{frame}


\begin{frame}[t]{700 hPa 일기도}
	\begin{tabular}{ll}
		\begin{minipage}[t]{0.3\textwidth}\scriptsize
			\begin{figure}[t]
				\includegraphics[trim=250 410 50 50, clip, 
				page=349, width=\textwidth]{\bookfile}
			\end{figure}
		\end{minipage}	
		&
		\begin{minipage}[t]{0.65\textwidth} \scriptsize	
			\begin{itemize}
				\item 700 hPa면은 약 3 km 상공의 등압면에 해당하며, 대류권 하부와 중부의 중간에 위치하여 대류권 하부를 대표하는 위치임.
				\item 등고선은 3,000 m를 기준으로 60 m 간격으로 그리며, 등온선은 5 ℃ 간격으로 그림
				\item 기온과 이슬점 차이가 4 ℃ 이내인 경우의 습윤 구역을 분석
				\item 700 hPa면에서의 고기압과 저기압은 폐곡선을 이루기보다는 기압골이나 기압마루 등의 편서풍파로 나타남
				\item 지상의 뇌우는 이 고도에서의 풍속과 같은 속도로 이동하는 경향이 있어서 뇌우의 이동을 예측할 때 700 hPa 고도의 바람이 이용됨
				\item 강한 고기압 중심의 하강기류로 인하여 700 hPa 고도의 기온이 (경험적으로 볼 때) 14℃ 이상이면, 하층으로부터의 상승기류를 저지하여 키 큰 적란운 발생을 억제. 즉, 뇌우의 발달을 억제함. 
				\item 700 hPa면의 기압골 전방의 온난기류와 후방의 한랭기류가 현저할 때, 그 기압골에 대응되는 지상 저기압이 발달함.
					
			\end{itemize}

		\end{minipage}
	\end{tabular}
\end{frame}



\begin{frame}[t]{700 hPa 일기도}
	\begin{tabular}{ll}
		\begin{minipage}[t]{0.3\textwidth}\scriptsize
			\begin{figure}[t]
				\includegraphics[trim=250 410 50 50, clip, 
				page=349, width=\textwidth]{\bookfile}
			\end{figure}
		\end{minipage}	
		&
		\begin{minipage}[t]{0.65\textwidth} \scriptsize	
			\begin{itemize}
				\item 일기도 삽입
					
			\end{itemize}

		\end{minipage}
	\end{tabular}
\end{frame}


\begin{frame}[t]{500 hPa 일기도}
	\begin{tabular}{ll}
		\begin{minipage}[t]{0.9\textwidth}\scriptsize
			\begin{figure}[t]
				\includegraphics[trim=50 50 50 550, clip, 
				page=363, width=\textwidth]{\bookfile}
			\end{figure}
		\end{minipage}	
		&
		\begin{minipage}[t]{0.05\textwidth} \scriptsize	
			
		\end{minipage}
	\end{tabular}
\end{frame}


\begin{frame}[t]{500 hPa 일기도}
	\begin{tabular}{ll}
		\begin{minipage}[t]{0.3\textwidth}\scriptsize
			\begin{figure}[t]
				\includegraphics[trim=250 410 50 50, clip, 
				page=349, width=\textwidth]{\bookfile}
			\end{figure}
		\end{minipage}	
		&
		\begin{minipage}[t]{0.65\textwidth} \scriptsize	
			\begin{itemize}
				\item 지상의 고기압 및 저기압, 전선, 태풍, 호우 등과 밀접한 관계를 가지고 있어서 응용 범위가 매우 넓은 일기도임.
				\item 지상 저기압이 500 hPa면에 기압골 전방에 있을 때는 발달하며, 후방에 있을 때는 약화됨.
				\item 500 hPa 기압골 전방은 상층에서 발산, 하층에는 수렴이 있어서 상승류가 잘 발달되며, 구름이 많고 많은 강수가 내림. 반면 후방에서는 상층 수렴, 하층 발산이 있어서 하강기류가 발달하므로 날씨가 좋음
				\item 지상의 저기압성 폭풍의 이동을 추정하는 데 매우 유용함. 
				\item 이동 방향이 500 hPa의 풍향과 동일하며, 이동속도는 500 hPa의 풍속의 1/4~1/2 정도임
				
			\end{itemize}

		\end{minipage}
	\end{tabular}
\end{frame}



\begin{frame}[t]{상층 일기도와 지상 일기도}
	\begin{tabular}{ll}
		\begin{minipage}[t]{0.8\textwidth}\scriptsize
			\begin{figure}[t]
				\includegraphics[trim=50 380 50 50, clip, 
				page=364, width=0.9\textwidth]{\bookfile}
			\end{figure}
		\end{minipage}	
		&
		\begin{minipage}[t]{0.05\textwidth} \scriptsize	
			\questionset{상층의 흐름과 지상의 날씨 관계를 설명하시오.}
			\solutionset{기압골과 연관되어 강한 사이클로닉 스톰을 나타냄}
		\end{minipage}
	\end{tabular}
\end{frame}


\begin{frame}[t]{300, 200 hPa 일기도}
	\begin{tabular}{ll}
		\begin{minipage}[t]{0.3\textwidth}\scriptsize
			\begin{figure}[t]
				\includegraphics[trim=250 410 50 50, clip, 
				page=349, width=\textwidth]{\bookfile}
			\end{figure}
		\end{minipage}	
		&
		\begin{minipage}[t]{0.65\textwidth} \scriptsize	
			\begin{itemize}
				\item 300 hPa면은 약 9~10 km 상공의 등압면이며, 200 hPa면은 약 11~12 km 상공의 등압면.
				\item 300 hPa면 일기도는 9,180 m를 기준으로 120 m 간격으로 그리며, 200 hPa면 일기도는 11,760 m를 기준으로 120 m 간격으로 그림.
				등온선은 둘 다 0 ℃ 기준으로  5 ℃ 간격으로 그림. 
				\item 이 고도에서는 제트기류가 가장 잘 나타남. 
				\item 겨울에는 고도가 비교적 낮아지고 여름에는 높아지기 때문에, 겨울과 같이 추운 계절에는 300 hPa 고도 근처에, 온난한 계절에는 200 hPa 고도 근처에서 가장 잘 나타남. 
			    \item 이 고도의 일기도에는 등풍속선이 표출된다. 
				\item 160 km/h 이상의 풍속이 매우 빠른 곳은 색깔로 표시할 수 있으며, 제트기류의 속도가 큰 부분을 제트 스트리크 라고 부른다. 
				\item 제트 스트리크 영역으로 진입하면 속도가 높아지며, 빠져나오면 속도가 늦춰진다. 
				\item 그러므로 이 영역으로 진입하면 발산, 빠져나오면 수렴이 발생한다.
				\item 즉, 상층의 발산을 하층의 상승기류를 유도하므로 하층의 수렴과 저기압의 발달로 이어지며, 상층의 수렴은 저기압의 약화를 가져온다. 
			    \item 악뇌우는 제트기류 근처에서 발달하는 경향이 있으므로, 제트기류의 위치를 파악하여 악기상을 예측(발달, 소멸 등) 하는데 사용
       			\item 이런 뇌우의 상단은 하단에 비해 풍속이 2 ~ 3배나 빠르므로 뇌우가 발달할수록 옆으로 기울어진다. 결국 상승기류와 하강기류의 중심이 서로 어긋나서 두 기류가 상쇄될 수 없게 되어 폭풍이 크게 성장한다. 
	
			\end{itemize}

		\end{minipage}
	\end{tabular}
\end{frame}



\begin{frame}[t]{300 hPa 일기도}
	\begin{tabular}{ll}
		\begin{minipage}[t]{0.3\textwidth}\scriptsize
			\begin{figure}[t]
				\includegraphics[trim=250 410 50 50, clip, 
				page=349, width=\textwidth]{\bookfile}
			\end{figure}
		\end{minipage}	
		&
		\begin{minipage}[t]{0.65\textwidth} \scriptsize	
			\questionset{겨울철 한대 전선 제트기류를 관측하는 데는 어느 고도(hPa)의 기상 자료가 가장 유용한가?}
			\solutionset{겨울철은 제트기류의 고도가 낮아 300 hPa 일기도가 유용하다.}

			\questionset{겨울철에 만일 제트기류가 관측자의 남쪽에 위치한다면, 관측자가 있는 지점의 기온은 평년에 비해 어떻게 될까?}
			\solutionset{열대 지역의 따뜻한 공기가 차고 건조한 극지방의 공기와 만나는 곳의 경계가 한대 전선 제트기류가 존재하는 곳인데, 제트기류가 관측자의 남쪽에 위치하고 있다면 극지방의 차갑고 건조한 공기로 인해 관측자의 기온은 평년보다 낮을 것이다. }

			\questionset{동서 패턴(Zonal)과 남북 패턴(Meridional)의 차이를 설명하시오.}
			\solutionset{편서풍이 서에서 동으로 거의 위도선에 나란하게 부는 경우를 동서 패턴이라고 함.
				이 지역의 폭풍은 (특히 겨울철에는) 매우 빠르게 동쪽으로 이동하고 날씨 변화가 급격히 일어나 가벼운 강수와 맑은 날씨가 번갈아 나타나게 됨.
				상층의 흐름은 남북으로 크게 사행하는 대규모의 기압골과 기압마루로 구성되는 경우가 많은데 이럴 때의 바람을 남북 패턴 이라고 함.
				이 경우 기상 패턴은 서에서 동으로 매우 천천히 이동하는데 때로는 정체하거나 반대 방향으로 이동하기도 함.
				※ 진폭이 큰 남북류가 발달하게 되면 기상 이변이 발달할 가능성이 있음. 
				- 강한 기압골: 겨울철 대설, 여름철 위력적인 뇌우와 토네이도의 발생 원인
				- 강한 기압마루: 여름철 폭염, 겨울철 온화한 기온 유지
				※ 남북으로 심하게 사행하는 패턴이 어느 지역에 정체하게 되면 지상 일기 패턴의 일변화를 매우 적게 만들기도 함. }

		\end{minipage}
	\end{tabular}
\end{frame}


\section{현대적 일기예보}



\begin{frame}[t]{수치 일기 예보}
	\begin{tabular}{ll}
		\begin{minipage}[t]{0.6\textwidth}\scriptsize
			\questionset{수치 일기 예보(Numerical Weather Prediction)란 무엇인가?}
			\solutionset{대기의 변화 과정을 계산하기 위하여 개발된 수치 모델을 사용하여 예보하는 기술
				실제 대기 상태 및 변화를 나타내는 정밀한 수학적 모델 사용
				방정식을 응용하는 방법이나 사용하는 모수가 다르며, 예보 대상이 되는 공간 영역도 다름}
		\end{minipage}	
		&
		\begin{minipage}[t]{0.35\textwidth} \scriptsize	
			\questionset{수치 일기 예보 과정에 대해 간단히 설명하시오.}
			\solutionset{초기값 입력: 수치예보 과정은 현재의 기상 상태(온도, 바람, 습도, 기압)을 컴퓨터 시뮬레이션의 초기값으로 입력하는 것으로부터 시작함. 
				특정시간 간격 결과값 도출: 입력 값에 대한 5분 또는 10분 후의 예측 값이 나오면 이 값을 이용하여 예측값을 지속적으로 만들어냄. 특정시간 간격의 시뮬레이션 결과값이 예보관에게 전달됨. 이렇게 만들어진 일기도는 미래 대기 상태를 예측하기 때문에 예상 차트라고 한다.
				MOS 과정: 수치 모델의 결과가 나오면 바로 전의 수치 예보 결과와 관측 자료를 비교하여 예보 자료의 정확성을 평가하여 수치 모델의 결과를 수정하는 MOS(Model Output Statics) 과정을 거침
				예보관 검토 및 수정: MOS 과정을 거친 후 모델의 단점과 기상학에 대한 풍부한 지식을 가진 예보관이 검토를 거쳐 수정함. 예보관은 여러 모델의 결과 중 경험에 비춰 모델을 선택하거나 모델의 결과를 종합함. 또 수치 모델 결과에는 나타나지 않는 뇌우나 미세규모 돌풍에 대해서 위성이나 기상 레이더 결과를 활용하여 기상 정보를 추가한다.}

		\end{minipage}
	\end{tabular}
\end{frame}



\begin{frame}[t]{수치 일기 예보}
	\begin{tabular}{ll}
		\begin{minipage}[t]{0.6\textwidth}\scriptsize
			\begin{figure}[t]
				\includegraphics[trim=50 450 250 50, clip, 
				page=367, width=\textwidth]{\bookfile}
			\end{figure}
		\end{minipage}	
		&
		\begin{minipage}[t]{0.35\textwidth} \scriptsize	
			\questionset{수치 일기 예보 장점에 대해 설명하시오.}
			\solutionset{복잡한 이론과 방대한 관측자료를 효과적으로 활용
				인간의 사고 체계로는 계산이 불가능한 복잡한 대기 현상도 대기의 운동과 현상을 지배하는 방정식들에 기반한 수치 예보 모델을 이용하여 해석할 수 있음}

		\end{minipage}
	\end{tabular}
\end{frame}



\begin{frame}[t]{수치 일기 예보}
	\begin{tabular}{ll}
		\begin{minipage}[t]{0.475\textwidth}\scriptsize
			\questionset{수치 일기 예보 모델의 예보 오차의 원인을 설명하시오.}
			\solutionset{1) 대기의 물리적 과정에 대한 부적절한 표현
				수치 해석에 사용한 지배 방정식은 복잡한 대기의 변화 과정을 간략화 한 것일 뿐,  특히 지표에서 일어나는 일이나 지형에 의한 효과들이 무시됨
				2) 초기 관측 자료에 포함된 오차
				아주 작은 초기 관측치의 오차도 ‘나비 효과’처럼 시간에 따라 증폭되기 때문
				3) 소규모 현상 모의의 어려움
				규모가 작은 현상은 예측하기가 매우 어렵기 때문
				4) 불충분한 모델의 해상도
				아주 작은 규모의 대기 현상도 대기의 변화에 영향을 주는데 수평 해상도를 증가시키기 위해서는 관측값도 정밀해지고 많아져야 하고, 컴퓨터의 성능도 향상되어야 함}

		\end{minipage}	
		&
		\begin{minipage}[t]{0.475\textwidth} \scriptsize	
			\questionset{수치 일기 예보의 한계에 대한 설명하시오.}
			\solutionset{1) 기술적인 한계성
				예보 모델을 운용할 초기 조건을 완벽하게 작성할 수 없음.
				관측에서 항상 존재하는 오차의 존재.
				예보 모델이 아직 완벽하게 만들어지지 못함.
				컴퓨터 능력의 유한성.
				2) 본질적인 한계성
				결정론적인 예보방정식을 사용하는데, 결정론은 이론일뿐 실제와는 다름.
				예보방정식에 이류 항이라는 비선형 항이 포함되어 있어 문제를 복잡하게 만듬.}
		\end{minipage}
	\end{tabular}
\end{frame}



\begin{frame}[t]{앙상블 예보(ensemble forecasting)}
	\begin{tabular}{ll}
		\begin{minipage}[t]{0.6\textwidth}\scriptsize
			\questionset{앙상블 예보(ensemble forecasting)란 무엇인지 설명하시오.}
			\solutionset{대기의 초기 조건이 조금만 달라져도 기상 패턴에 아주 큰 영향을 미칠 수 있는데, 대기의 이러한 카오스적 거동을 다루기 위해 앙상블 예보(ensemble forecasting) 기법을 사용함.
				관측 오차의 크기에 해당하는 정도의 초기 조건의 차이를 주어 여러 가지 예보를 만든 후 이러한 결과가 예보에 어떠한 영향을 미치는 지 평가(단일모델앙상블)함. 즉 예보의 불확실성에 대한 정보를 제공하기 위해 사용함.
				또한, 여러 개의 다른 수치예측모델을 이용하여 예보를 만들어 이를 활용하는 방법(다중모델앙상블)도 있음.
				예를 들어 예상 일기도에서 특정 지역에 24시간 동안 강수가 있다고 나타났는데, 초기 조건을 달리하여 여러 번의 계산 결과, 대부분의 예상 일기도에서 그 지역의 강수를 예보했다면 실제로 이러한 일이 일어날 가능성이 매우 높고 이 예보는 신빙성이 높다는 것을 의미함.}
				
		\end{minipage}	
		&
		\begin{minipage}[t]{0.35\textwidth} \scriptsize	
			\begin{itemize}
				\item 시간에 따라 예보 모델 간의 예보 오차가 크게 차이가 남
				\item 긴 시간 동안 예측성의 유지와 함께 예측의 불확실성에 대한 정보도 함께 제공
				\item 다양한 예측 가능 시나리오를 제공하므로, 예측 기간 동안의 재해 발생 가능성과 재해의 종류에 대한 정보를 제공. 
				\item 즉, 사전 대비를 위한 정보를 제공.
					
			\end{itemize}

		\end{minipage}
	\end{tabular}
\end{frame}


\section{다른 일기예보 방법}

\begin{frame}[t]{지속성 예보}
	\begin{tabular}{ll}
		\begin{minipage}[t]{0.6\textwidth}\scriptsize
			\questionset{지속성 예보의 정의와 한계를 설명하시오. }
			\solutionset{일반적으로 특정 기상 조건은 수 시간 또는 수 일 동안 변하지 않고 지속될 수 있음. 지속성 예보란 특정 기상의 지속성을 근거로 하여 동일한 기상이 수 시간 혹은 수 일동안 지속될 것으로 예측하는 것을 의미함. 
				예를 들어 특정 지역이 고기압에 속해있다면 6시간에서 12시간 동안은 맑은 날씨가 지속될 것이라고 예보하는 것이 한 가지 예.
				지속성 예보는 기상 상태의 강도나 이동 방향의 변화를 고려하지 않고, 요란의 생성 또는 소멸도 고려하지 않음. }

		\end{minipage}	
		&
		\begin{minipage}[t]{0.35\textwidth} \scriptsize	
			\begin{itemize}
				\item 				
				\item 
				\item 
				
			\end{itemize}

		\end{minipage}
	\end{tabular}
\end{frame}



\begin{frame}[t]{기후적 예보}
	\begin{tabular}{ll}
		\begin{minipage}[t]{0.5\textwidth}\scriptsize
			\begin{figure}[t]
				\includegraphics[trim=350 50 30 500, clip, 
				page=368, width=\textwidth]{\bookfile}
			\end{figure}
		\end{minipage}	
		&
		\begin{minipage}[t]{0.45\textwidth} \scriptsize	
			\questionset{장기간 축적된 기상 자료에 근거하여 예보하는 방법을 설명하시오.}
			\solutionset{장기간 축적된 기상 자료에 근거하는 예보방법을 기후적 예보라고 부름.
				기후적 예보는 보통 농업 기상 예보에 유용한데, 특정 지역의 5월 맑은 날씨가 30일 중 27일이었다면 5월의 날씨가 맑을 것이라고 예보하는 식. 
				기후적 예보는 화이트크리스마스와 같은 특별한 날의 날씨 예측에도 사용함.}
		\end{minipage}
	\end{tabular}
\end{frame}



\begin{frame}[t]{패턴 인식법(유추법)과 경향 예보}
	\begin{tabular}{ll}
		\begin{minipage}[t]{0.475\textwidth}\scriptsize
			\questionset{패턴 인식법이라고 부르는 통계적 방법에 대해 설명하시오.}
			\solutionset{이 방법은 컴퓨터 모델링이 발달하기 전에 가장 많이 사용하던 일기 예보법으로, 일반적으로 날씨의 변화가 반복되어 나타난다는 가정 하에, 현재 기상 패턴과 가장 
				유사한 과거의 일기도를 찾아 비교한 후 현재의 기상 상태가 어떻게 변화할 지를 예보하는 것. 
				현재에도 단기 수치 예보를 향상 시키기 위한 일환으로 사용하고 있음.}
		\end{minipage}	
		&
		\begin{minipage}[t]{0.475\textwidth} \scriptsize	
			\questionset{경향 예보(trend forecasting)란 무엇인지 설명하시오.}
			\solutionset{경향예보란 전선, 태풍, 구름, 강수 등의 현상에 대한 이동 방향과 속도를 결정하여 이러한 현상의 미래 위치를 추정한 후 일기를 예보하는 방법.
				주로 수명이 짧은 기상 현상(우박, 토네이도 등) 예보에 잘 적용이 됨. (수 시간 단위의 짧은 기간 예보에만 유효)}
		\end{minipage}
	\end{tabular}
\end{frame}



\begin{frame}[t]{실황 예보}
	\begin{tabular}{ll}
		\begin{minipage}[t]{0.6\textwidth}\scriptsize
			\begin{figure}[t]
				\includegraphics[trim=250 410 50 50, clip, 
				page=349, width=\textwidth]{\bookfile}
			\end{figure}
		\end{minipage}	
		&
		\begin{minipage}[t]{0.35\textwidth} \scriptsize	
			\questionset{기상 레이더와 위성 영상에 크게 의존하는 초단기예보를 무엇이라 하는가?}
			\solutionset{실황 예보(nowcasting)라 부르며, 대기의 상태가 급격하게 변하지 않는다는 속성을 이용한 것임
				토네이도, 우박과 같은 현상에 대한 경보는 신속히 이루어져야 하는데 기상레이더나 기상위성과 같은 원격 탐사 장비를 이용하여 얻은 자료를 신속히 처리하여 이동할 지점을 구체적으로 예측하는 초단기 예보를 특히 실황예보라고 함.}

			\questionset{실황 예보 기술을 사용하여 예보하는 현상에는 어떠한 것들이 있는가?}
			\solutionset{우박, 토네이도, 미규모의 돌풍 등}

		\end{minipage}
	\end{tabular}
\end{frame}







\section{기상위성: 예보의 도구}



\begin{frame}[t]{기상위성의 궤도}
	\begin{tabular}{ll}
		\begin{minipage}[t]{0.35\textwidth}\scriptsize
			\begin{figure}[t]
				\includegraphics[trim=350 330 40 120, clip, 
				page=370, width=\textwidth]{\bookfile}
			\end{figure}
		\end{minipage}	
		&
		\begin{minipage}[t]{0.6\textwidth} \scriptsize	
			\questionset{극궤도 위성과 정지궤도 위성의 특징을 비교하여 설명하시오.}
			\solutionset{
				% Please add the following required packages to your document preamble:
				% \usepackage{graphicx}
				\begin{table}[]
					\resizebox{\textwidth}{!}{%
					\begin{tabular}{c|c|c}
					\textbf{정지궤도 위성(GOES)}      & \textbf{구분} & \textbf{극궤도 위성(POES)}   \\
					약 36,000km                  & 고도          & 850km                   \\
					지구 자전 각속도와 같은 각속도로 궤도 운동을 함 & 공전속도        & 100분에 지구 한 번 공전         \\
					특정 지점을 지속 관측 가능             & 관측 지역       & 많은 지역을 촬영 가능            \\
					고도가 높아 해상도는 극궤도 위성에 비해 낮음   & 해상도         & 지표 가까이 운행하므로 촬영 해상도가 높음 \\
					통신위성, 기상위성                  & 용도          & 정찰 위성                  
					\end{tabular}%
					}
					\end{table}
				}

		\end{minipage}
	\end{tabular}
\end{frame}



\begin{frame}[t]{가시광선 영상(400~700 nm)}
	\begin{itemize}\scriptsize
		\item 구름표면(또는 운정)이나 지표면으로부터 반사된 태양광선의 강도를 기록
		\item 반사도가 높을수록 밝게 표현됨 : 눈, 얼음, 구름이 있는 지역은 흰색,육지는 보통 회색, 바다는 거의 검정 (밝기 : 구름>육지>바다)
		\item 구름의 두께, 모양, 구름의 집단에 대한 정보를 제공 : 두께가 두꺼우면 반사율이 높아 영상이 밝게 나옴  (보통 고도가 높은 구름일수록 반사율이 높아 영상이 밝게 나옴)
			: 입자가 큰 구름이 입자가 작은 구름보다 영상이 밝게 나옴
		\item 가시 영상은 낮에만 사용 가능하며, 적외 영상 보다는 고해상도의 영상을 제공함. 
	\end{itemize}
	\begin{tabular}{ll}
		\begin{minipage}[t]{0.94\textwidth}\scriptsize
			\begin{figure}[t]
				\includegraphics[trim=50 50 50 550, clip, 
				page=371, width=\textwidth]{\bookfile}
			\end{figure}
		\end{minipage}	
		&
		\begin{minipage}[t]{0.01\textwidth} \scriptsize	


		\end{minipage}
	\end{tabular}
\end{frame}



\begin{frame}[t]{적외선 영상(10.5~12.5μm)}
	\begin{tabular}{ll}
		\begin{minipage}[t]{0.4\textwidth}\scriptsize
			\begin{figure}[t]
				\includegraphics[trim=50 410 350 50, clip, 
				page=372, width=\textwidth]{\bookfile}
			\end{figure}
		\end{minipage}	
		&
		\begin{minipage}[t]{0.55\textwidth} \scriptsize	
			\begin{itemize}
				\item 물체에서 복사(반사가 아님)되는 전자기파에서 얻는 것으로 복사에너지가 위성에 감지되는 것 (모든 물체는 물체 온도의 네 제곱에 비례하는 복사에너지 방출)
				\item 어느 구름이 비를 내리고, 폭풍 동반 여부를 판단하는데 유용
				\item 따뜻한 물체는 어둡게 나타나고 차가운 물체는 밝게 나타남 (보통 상층운이 밝게, 하층운이 어둡게 나타남)
    			\item 적외 영상은 낮과 밤 모두 사용 가능하며, 구름 꼭대기의 온도 정보를 제공함. 
			 	
			\end{itemize}

		\end{minipage}
	\end{tabular}
\end{frame}



\begin{frame}[t]{수증기 영상}
	\begin{tabular}{ll}
		\begin{minipage}[t]{0.6\textwidth}\scriptsize
			\begin{figure}[t]
				\includegraphics[trim=350 50 50 450, clip, 
				page=372, width=\textwidth]{\bookfile}
			\end{figure}
		\end{minipage}	
		&
		\begin{minipage}[t]{0.35\textwidth} \scriptsize	
			\questionset{수증기 영상을 통하여 얻을 수 있는 정보는 무엇인가?}
			\solutionset{수증기는 지구 복사 중 파장 6.7㎛ 영역을 잘 흡수한다. 
				따라서 해당 영역의 탐지기를 장착한 위성은 대기의 수증기 농도 분포를 파악할 수 있다. 
				복사가 잘 나타나는 영역은 수증기가 적은 영역이고, 복사가 적게 나타나는 영역은 수증기가 많은 영역인데 영상에서는 밝게 나타난다(반전). 
				대부분의 전선은 수증기 함량이 크게 다른 두 기단의 경계에서 나타나므로 수증기 영상은 전선의 경계를 파악하는데 중요하게 쓰인다.}

		\end{minipage}
	\end{tabular}
\end{frame}



\begin{frame}[t]{위성 영상의 해석}
	\begin{tabular}{ll}
		\begin{minipage}[t]{0.6\textwidth}\scriptsize
			\begin{figure}[t]
				\includegraphics[trim=40 410 50 100, clip, 
				page=373, width=\textwidth]{\bookfile}
			\end{figure}
		\end{minipage}	
		&
		\begin{minipage}[t]{0.35\textwidth} \scriptsize	
			\questionset{1. On which image (A or B) are the cloud shapes and patterns more easily recognized?}
			
			\questionset{2. Which one of these images was obtained using infrared imagery? How can you tell?}
			
			\questionset{3. What types of clouds are found in the western Great Plains (just east of the Rockies)?}
			
			\questionset{4. Are the clouds over the eastern Pacific Ocean mainly high clouds or middle- to low-level clouds?}

		\end{minipage}
	\end{tabular}
\end{frame}



\begin{frame}[t]{}
	\begin{tabular}{ll}
		\begin{minipage}[t]{0.6\textwidth}\scriptsize
			\begin{figure}[t]
				\includegraphics[trim=250 410 50 50, clip, 
				page=349, width=\textwidth]{\bookfile}
			\end{figure}
		\end{minipage}	
		&
		\begin{minipage}[t]{0.35\textwidth} \scriptsize	
			\begin{itemize}
				\item 				
				\item 
				\item 
				
			\end{itemize}

		\end{minipage}
	\end{tabular}
\end{frame}



\begin{frame}[t]{}
	\begin{tabular}{ll}
		\begin{minipage}[t]{0.475\textwidth}\scriptsize
			\questionset{기상 위성 영상에서 비를 내릴 가능성이 높은 구름을 식별하는 원리를 설명하시오.}
			\solutionset{적외 영상과 가시 영상에서 모두 흰색으로 나타나는 부분이 비가 내릴 가능성이 높은 구름이다.
				적외 영상은 지표로부터 멀리 떨어져 있어 복사온도가 낮으면 흰색으로 나타난다.
				적란운은 구름 상부의 온도가 낮아 적외 영상으로 매우 흰색으로 나타나게 되며, 적란운인지 확실히 확인하기 위해서는 반사도를 측정하여 구름의 두께를 추정할 수 있는 가시 영상에서도 매우 흰색으로 나타나는 지 확인해야 한다.}

		\end{minipage}	
		&
		\begin{minipage}[t]{0.475\textwidth} \scriptsize	
			\questionset{위성 영상에서 상층운과 하층운을 구별하는 원리를 설명하시오.}
			\solutionset{가시 영상이 적외선 영상보다 밝게 보이면 하층운이며, 흐리고 강수의 가능성이 있다.
				반면, 가시 영상이 적외선 영상보다 어두우면 상층운이며, 맑거나 갤 가능성이 크다.}

		\end{minipage}
	\end{tabular}
\end{frame}



\section{예보의 종류}



\begin{frame}[t]{정량적 예보, 정성적 예보}
	\begin{tabular}{ll}
		\begin{minipage}[t]{0.5\textwidth}\scriptsize
			\begin{figure}[t]
				\includegraphics[trim=50 410 280 50, clip, 
				page=374, width=\textwidth]{\bookfile}
			\end{figure}
		\end{minipage}	
		&
		\begin{minipage}[t]{0.35\textwidth} \scriptsize	
			\questionset{정량적 예보와 정성적 예보를 비교하시오.}
			\solutionset{정성적 예보는 관측은 가능하나 정확한 양으로 나타내기 어려운 요소를 다룸. 이러한 예보는 정확성을 평가하기는 어렵지만, 일반 대중에게는 유용함
				정량적 예보는 양을 측정할 수 있는 요소를 다룸. 
				예를 들어 최고 기온과 최저 기온 예보, 일정 기간 동안의 강수량 예보 등}
		\end{minipage}
	\end{tabular}
\end{frame}



\begin{frame}[t]{강수 확률 예보}
	\begin{tabular}{ll}
		\begin{minipage}[t]{0.45\textwidth}\scriptsize
			\begin{figure}[t]
				\includegraphics[trim=40 480 260 50, clip, 
				page=375, width=\textwidth]{\bookfile}
			\end{figure}						

		\end{minipage}	
		&
		\begin{minipage}[t]{0.5\textwidth} \scriptsize	
			퍼센트 확률로 제공되는 것으로 ‘뇌우의 확률은 40\%’와 같이 제공되는 것을 강수 확률예보(Probability of Precipitation, PoP) 라고 부름

			$$
			\mathrm{PoP}=(C \times A) \times 100 \%
			$$

			C는 예보 대상 지역의 어느 지점인가에 강수가 내릴 확률,  A는 비가 온다면 예보 대상 면적 중 강수가 내릴 면적의 퍼센트를 의미함

			예보 대상 지역에 50\%의 강수 확률이 있고, 전체 면적의 80\%에 해당하는 지역에서 강수가 내린다고 하면, 

			$$
			\mathrm{PoP}=(0.5 \times 0.8) \times 100 \%=40 \%
			$$

			예보 대상 지역의 어느 지점에서도 강수 확률은 40\%를 의미. 평균적으로 봤을 때 강수가 있을 확률이 40\%
		
		\end{minipage}
	\end{tabular}
\end{frame}



\begin{frame}[t]{확률 예보}
	\begin{tabular}{ll}
		\begin{minipage}[t]{0.5\textwidth}\scriptsize
			\begin{figure}[t]
				\includegraphics[trim=40 230 260 320, clip, 
				page=375, width=\textwidth]{\bookfile}
			\end{figure}						
		\end{minipage}	
		&
		\begin{minipage}[t]{0.35\textwidth} \scriptsize	
			\begin{itemize}
				\item 				
				\item 
				\item 
				
			\end{itemize}

		\end{minipage}
	\end{tabular}
\end{frame}



\begin{frame}[t]{단기, 중기, 장기 예보}
	\begin{tabular}{ll}
		\begin{minipage}[t]{0.6\textwidth}\scriptsize
			\begin{itemize}
				\item 단기 예보 : 48시간(2일) 예보
					단기 예보는 주로 수치예보에 의존하며, 평균적인 기후 조건이나 미기후적인 특성, 지리적 특성도 잘 파악해야 단기 예보의 정확성을 높을 수 있음
	
				\item 중기 예보 : 3~7일 예보
    				중기 예보는 단계 예보에 비해 정확도가 떨어지는 편이지만, 최근에는 정확성이 향상되어서 10일 예보까지도 가능할 정도임
				
				\item 장기 예보 : 7일보다 먼 미래에 대한 예보
    				장기예보는 수치예보와 통계예보를 절충하여 사용. 다만 30일 이상을 전망하는 경우에는 습도와 온도가 예년보다 높을지, 낮을지와 같은 개략적 정보만 예보함. 
					
			\end{itemize}
		\end{minipage}	
		&
		\begin{minipage}[t]{0.35\textwidth} \scriptsize	
			

		\end{minipage}
	\end{tabular}
\end{frame}



\begin{frame}[t]{장기 예보}
	\begin{tabular}{ll}
		\begin{minipage}[t]{0.6\textwidth}\scriptsize
			\begin{figure}[t]
				\includegraphics[trim=50 410 50 50, clip, 
				page=376, width=\textwidth]{\bookfile}
			\end{figure}
		\end{minipage}	
		&
		\begin{minipage}[t]{0.35\textwidth} \scriptsize	
			\questionset{장기 예보에서 활용하는 정보는 무엇인가?}
			\solutionset{1) 각 지역 기상 요소의 30년 평균치
				2) 눈 또는 얼음이 덮고 있는 지표 면적, 여름 동안의 지속적인 토양의 수분 함유 상태
				3) 현재의 기온과 강수 상태
				4) 해수면 온도(엘니뇨, 라니냐)와 상층 대기 흐름}
		\end{minipage}
	\end{tabular}
\end{frame}





\section{예보자의 역할}



\begin{frame}[t]{고급 대화식 기상 처리 시스템}
	\begin{tabular}{ll}
		\begin{minipage}[t]{0.5\textwidth}\scriptsize
			\begin{figure}[t]
				\includegraphics[trim=0 0 250 460, clip, 
				page=377, width=\textwidth]{\bookfile}
			\end{figure}
		\end{minipage}	
		&
		\begin{minipage}[t]{0.45\textwidth} \scriptsize	
			\questionset{고급 대화식 기상처리시스템(Advanced Weather Interactive Processing System, AWIPS)에 대해 설명하시오.}
			\solutionset{AWIPS는 정보의 처리, 표출, 통신 시스템으로서 모든 모델 자료 및 위성과 레이더 등의 관측 자료를 한곳으로 수집한다. 3개의 그래픽 표출 모니터는 워크스테이션 본체와 연결되어 있으며 적어도 20개의 그래픽 창을 가지고 있는데 모두 다른 기상 정보를 나타내 준다.}


		\end{minipage}
	\end{tabular}
\end{frame}



\begin{frame}[t]{그래픽 예보}
	\begin{tabular}{ll}
		\begin{minipage}[t]{0.94\textwidth}\scriptsize
			\begin{figure}[t]
				\includegraphics[trim=50 470 50 50, clip, 
				page=378, width=\textwidth]{\bookfile}
			\end{figure}
		\end{minipage}	
		&
		\begin{minipage}[t]{0.01\textwidth} \scriptsize	
			\begin{itemize}
				\item 				
				\item 
				\item 
				
			\end{itemize}

		\end{minipage}
	\end{tabular}
\end{frame}



\section{예보의 정확도}

\begin{frame}[t]{예보 스킬}
	\begin{tabular}{ll}
		\begin{minipage}[t]{0.6\textwidth}\scriptsize
			\begin{figure}[t]
				\includegraphics[trim=250 410 50 50, clip, 
				page=349, width=\textwidth]{\bookfile}
			\end{figure}
		\end{minipage}	
		&
		\begin{minipage}[t]{0.35\textwidth} \scriptsize	
			\questionset{일기 예보의 적중률이 예보 스킬을 평가하는데 항상 좋은 지표는 아니다. 그 이유를 설명하시오.}
			\solutionset{로스앤젤레스의 강수일수는 1년 중에 11일 밖에 되지 않는다. 
		`		따라서 내가 매일 비가 오지 않는다고 예보한다면 예보의 적중률은 97\%나 된다. 
				따라서 단순히 기후적인 평균치에 의해 예보한 것보다는 나은 예보를 해야 좋은 예보기술을 가지고 있다고 말할 수 있다. 
				즉, 이런 경우 비가 오는 날을 잘 맞추어야 예보 기술이 뛰어나다 평가할 수 있다. 
				확률적 예보가 이루어지는 유일한 기상 요소가 강수량인데, 
				강수예보는 같은 조건에서 얼마나 비가 자주 내렸는지를 알려 준다. 
				예를 들어, 어떤 지역에 0.25mm이 상의 비가 내릴 강수 확률이 70\%라면 같은 조건에서 비가 내린 경우가 10번 중 일곱 번이었다는 말이다.'}

		\end{minipage}
	\end{tabular}
\end{frame}



