%%%%%%%%%% *** The Title %%%%%%%%%%
\title[]{대기에 대한 소개\\\small{제1장}}

\begin{frame}[plain] %title page
	\titlepage
\end{frame}


\begin{frame}[plain] %title page
	\ccpage
\end{frame}


	
\section{대기에 대한 포커스}

\begin{frame}[t]{극한 일기의 발생}
	\begin{tabular}{ll}
		\begin{minipage}[t]{0.50\textwidth}\	
			\begin{figure}[t]
				\includegraphics[trim=0 0 360 520, clip, page=35, width=\textwidth]
				{\bookfile}
			\end{figure}
		\end{minipage}	
		&
		\begin{minipage}[t]{0.45\textwidth}
			\begin{itemize}
				\item 날씨는 매일 뉴스에 나오는 일상의 한 부분
				\item 기록적인 기상현상의 발생 강도와 빈도가 증가하고 있음
			\end{itemize}
		\end{minipage}
	\end{tabular}
\end{frame}





\begin{frame}[t]{인간과 대기}
	\begin{tabular}{ll}
		\begin{minipage}[t]{0.30\textwidth}\					\begin{figure}{}
				\includegraphics[trim=325 145 50 155, clip, page=35, width=\textwidth]
				{\bookfile}
			\end{figure}
		\end{minipage} 
		&
		\begin{minipage}[t]{0.65\textwidth}
			\begin{itemize}
				\item 날씨는 개인에게 직접적인 피해를 주기도 하고
				\item 농업, 에너지 사용, 수자원, 수송, 산업 등에 영향을 미친다.
				\item 또한 인간은 대기환경에 영향을 미치고 있다 
			\end{itemize}
		\end{minipage}
	\end{tabular}
\end{frame}


\begin{frame}[t]{기상학, 날씨, 기후}
	\begin{tabular}{ll}
		\begin{minipage}[t]{0.45\textwidth}\						\begin{figure}{}
				\includegraphics[trim=280 435 30 60, clip, page=36, width=\textwidth]{\bookfile}
			\end{figure}
		\end{minipage} 
	&
		\begin{minipage}[t]{0.50\textwidth}
			\begin{itemize}
				\item 기상학(meteorology): 대기와 날씨에 대해 과학적으로 연구하는 학문
				\item 날씨(weather): 계속적으로 변화하며, 때때로 시간마다 또는 날마다 변하는 주어진 시간과 공간에서의 대기상태
				\item 기후(climate): 수십 년에 걸처 누적된 관측으로부터 얻는 통계적인 날씨 정보의 조합 (평균값, 변동과 극값뿐 아니라 예외적 현상이 일어날 가능성도 포함) 
			\end{itemize}
		\end{minipage}		
	\end{tabular}
\end{frame}



\begin{frame}[t]{기상학, 날씨, 기후}
	\begin{tabular}{ll}
		\begin{minipage}[t]{0.63\textwidth}		
			\begin{figure}[t]
				\includegraphics[trim=35 500 340 55, clip, page=37, width=0.8\textwidth]{\bookfile}
			\end{figure}
			\begin{itemize}\scriptsize 
				\item 여러 종류의 자료로 부터 그 지역의 정보를 얻을 수 있다. 하지만 이 정보를 날씨를 예측할 수는 없다.  	
			\end{itemize}
		\end{minipage}
		&
		\begin{minipage}[t]{0.32\textwidth}
			\begin{figure}[t]
				\includegraphics[trim=355 350 80 65, clip, page=37, 
				width=\textwidth]{\bookfile}
			\end{figure}
		\end{minipage}
	\end{tabular}
\end{frame}





\begin{frame}[t]{기상재해: 기상 요소의 습격}
		\begin{tabular}{ll}
		\begin{minipage}[t]{0.66\textwidth}		
			\begin{figure}[t]
				\includegraphics[trim=355 30 30 550, clip, page=37, width=0.8\textwidth]{\bookfile}
			\end{figure}
			\begin{itemize}
				\item 많은 자연 재해가 대기와 관련되어 있다.
			\end{itemize}
		\end{minipage}
		&
		\begin{minipage}[t]{0.3\textwidth}
			\begin{figure}[]
				\includegraphics[trim=90 360 350 55, clip, page=38, width=\textwidth]{\bookfile} 
			\end{figure}	
		\end{minipage}
	\end{tabular}
\end{frame}





\section{과학적 탐구의 본성}


\begin{frame}[t]{과학적 방법}
	\begin{tabular}{ll}
		\begin{minipage}[t]{.95\textwidth}
			\begin{itemize}
				\item 자연 현상에 대한 질문
				\item 그 질문과 관련한 관측이나 측정을 통해 과학적 자료 모으기
				\item 그 자료와 관련한 질문을 던지고, 이 질문을 설명할 수 있는 한가지 이상의 적절한 가설 세우기
				\item 가설을 검증할 수 있는 관측, 실험, 모형 등을 세우기
				\item 엄격한 검증과정을 바탕으로 가설을 받아들이거나, 수정하거나, 기각하기
				\item 자료나 결과들을 비판적 의경과 추가적인 실험을 위해 과학계와 공유하기
			\end{itemize}
		\end{minipage}
		&
		
	\end{tabular}
\end{frame}





\begin{frame}[t]{관측과 측정}
	\begin{tabular}{ll}
		\begin{minipage}[t]{.60\textwidth}
			\begin{figure}
				\includegraphics[trim=345 515 5 0, clip, page=40, width=\textwidth]{\bookfile}
			\end{figure}
		\end{minipage}
		&
	\begin{minipage}[t]{0.35\textwidth}
		\begin{itemize}
			\scriptsize
			\item 교과서에는 자동기상관측시스템(Automated Surface Observing System) 표현되어 있음
			\item 보통 AWS라고 부르며, Automatic Weather Station 또는 Automatic Weather System 등으로 표현하며,
			\item 기온, 습도, 기압, 풍항, 풍속 등의 기상 요소들을 센서를 통해 그 값을 읽어 데이터 로거에 저장함.
			\item 통신과 인터넷의 발달로 자료를 바로 전송하기도 함.
			\item 우리나라 기상청에서도 많이 사용하고 있음. 
		\end{itemize}
	\end{minipage}		
	\end{tabular}
\end{frame}



\begin{frame}[t]{우주에서의 지구 감시}
	\begin{tabular}{ll}
		\begin{minipage}[t]{.95\textwidth}
			\begin{figure}
				\includegraphics[trim=20 230 30 320, clip, page=39, width=\textwidth]{\bookfile}
			\end{figure}
			\begin{itemize}	\scriptsize
				\item NASA’s Tropical Rainfall Measuring Mission (TRMM)
				\item 인공위성을 이용한 지구 관측 자료는 아주 유용하게 사용된다.
			\end{itemize}
		\end{minipage}
		&
		\begin{minipage}[t]{.03\textwidth}

		\end{minipage}
		
	\end{tabular}
\end{frame}





\section{시스템으로서의 지구}



\begin{frame}[t]{대기권}
	\begin{tabular}{ll}
		\begin{minipage}[t]{.55\textwidth}
			\centering
			\begin{figure}
			\includegraphics[trim=300 35 30 535, 
			clip, page=42, width=\textwidth]{\bookfile}
			\end{figure}
		\end{minipage}
		&
		\begin{minipage}[t]{.35\textwidth}
			\questionset{지구 반지름에 비해 대기의 두께는 얼마나 두꺼운가?}
			\solutionset{지구 반지름: $6370 \rm{~km}$,\\ 대류권: $10 \rm{~km}$ \newline}
		
			\questionset{야광운(Noctilucent clouds)이 무엇인가?}
			\solutionset{야광운(夜光雲)은 고위도지방 $70 \sim 90{^\circ}$ 부근 고도 $76 \sim 85\rm{~km}$ 높이의 중간권에 생기는 구름을 말한다. 약 120년 전에 처음 관측되었으나 그 높이에서 왜 생기는지는 아직 알 수 없다. \newline}
		\end{minipage}
	\end{tabular}
\end{frame}




\begin{frame}[t]{물의 행성}
	\begin{tabular}{ll}
		\begin{minipage}[t]{0.55\textwidth}
			\begin{figure}
				\includegraphics[trim=30 415 285 65, clip, page=43, width=\textwidth]{\bookfile}
			\end{figure}
		\end{minipage}
		&
		\begin{minipage}[t]{0.4\textwidth}
			\begin{itemize}
				\item 지구를 푸른 행성이라고 부름
				\item 지구 상에 존재하는 물 중에서 하천, 호수, 빙하, 지하수 및 대기 중의 수분은 아주 적지만, 생물권에 많은 영향을 미침
			\end{itemize}

		\end{minipage}		
	\end{tabular}
\end{frame}





\begin{frame}[t]{물 순환}
	\begin{tabular}{ll}
		\begin{minipage}[t]{.55\textwidth}
			\begin{figure}
				\includegraphics[trim=290 0 50 470, clip, page=44, width=\textwidth]{\bookfile}
			\end{figure}
		\end{minipage}
		&
		\begin{minipage}[t]{0.40\textwidth}
			\begin{itemize}
				\scriptsize 
				\item 물의 순환은 시작과 끝이 없음
				\item 바닷물은 증발(evaporation)되고, 
				얼음과 눈은 수증기로 바로 승화, 식물의 증발산(evapotranspiration)
				\item 대기 중의 수증기는 대기로 올라가며 여기서 온도가 차가워지면 구름에 응축
				\item 구름의 입자는 충돌하고 상승하다가 강수(precipitation) 현상을 일으킴
				\item 일부 강수는 눈으로 떨어지며 수천년에 걸쳐 언 물을 담을 수 있는 만년설이나 빙하
			\end{itemize}
		\end{minipage}		
	\end{tabular}
\end{frame}



\begin{frame}[t]{지구 시스템 과학}
	\begin{tabular}{ll}
		\begin{minipage}[t]{.5\textwidth}
			\begin{figure}
				\begin{tikzpicture}
					\node[anchor=south west,inner sep=0] (image) at (0,0) {\includegraphics[trim=170 40 50 390, clip, page=43, width=\textwidth]{\bookfile}};
					\begin{scope}[x={(image.south east)},y={(image.north west)}]
						\filldraw[fill=white, draw = white] (0,1) rectangle (0.25, 0.6);
						%\draw[help lines,xstep=.1,ystep=.1] (0,0) grid (1,1);
						%\foreach \x in {0,1,...,9} { \node [anchor=north] at (\x/10,0) {0.\x}; }
						%\foreach \y in {0,1,...,9} { \node [anchor=east] at (0,\y/10) {0.\y}; }
					\end{scope};
				\end{tikzpicture}
			\end{figure}
		\end{minipage}
		&
		\begin{minipage}[t]{.45\textwidth}
			\begin{itemize}
				\item 지구계(earth system)는 에너지 교환은 일어나지만, 물질 교환은 일어나지 않는 닫힌계
				\item 지권, 기권, 수권, 생물권 그리고 외권이 서로 상호작용하며 서로 다른 부분들을 종합적으로 접근하는 방식
			\end{itemize}
		\end{minipage}
	\end{tabular}
\end{frame}



\section{대기의 구성}

\begin{frame}[t]{대기의 구성}
	\begin{tabular}{ll}
		\begin{minipage}[t]{.40\textwidth}
			\begin{figure} 
				\includegraphics[trim=55 485 380 65, 
				clip, page=47, width=\textwidth]{\bookfile} 
			\end{figure}
		\end{minipage}
		&
		\begin{minipage}[t]{.55\textwidth}
			\begin{itemize}
				\item 건조 공기: 공기에서 수증기를 제외한 나머지 기체
				\item 대기 중 수증기 농도가 시공간적으로 변하기 때문에 보통 건조공기를 기준으로 대기의 조성을 나타냄. ($1 \sim 4 \%$)
			\end{itemize}
			\begin{enumerate}
				\item 영구 기체: 생명체와 화학반응에 중요 \\
					ex) 질소, 산소, (이산화탄소)
				\item 변량 기체: 기상현상에 중요 \\
					ex) 에어로졸, 오존 등
			\end{enumerate}	
			\questionset {대기의 평균 분자량을 계산하시오.}
			\solutionset{$(28 \times 0.7808) + (32 \times 0.2095) + (40 \times 0.0093) + (44 \times 0.0004 = 28.96$}
	\end{minipage}
	\end{tabular}
\end{frame}

\begin{frame}[t]{영구 기체}
	\begin{tabular}{ll}
		\begin{minipage}[t]{.95\textwidth}
			\begin{enumerate}
				\item 질소 : 토양 박테리아의 생물학적 작용으로 대기에서 소멸(질소고정).\\
				⇒ 동식물의 부패과정으로 대기로 돌아옴.
				\item 산소 : 유기물이 산화하거나 다른 원소와 결합하여 산화물 생성할 때 혹은 생물이 호흡할 때 대기에서 소멸.\\
				⇒ 식물의 광합성 작용을 통해 대기로 돌아옴.
				\item 이산화탄소 : 박테리아의 발효 및 탈질화 과정, 동식물 세포의 호기성 호흡 등 생물학적 기작으로 생성, 일산화탄소와 유기가스의 화학적 산화, 화산에서의 기체 방출, 화석연료 연소 등 자연적 및 인위적 생성.\\
				⇒ 광합성 반응, 지표수에의 용해, 화학적 부식 등으로 대기 중 제거
			\end{enumerate}
		\end{minipage}
		&		
	\end{tabular}
\end{frame}


\begin{frame}[t]{지구 대기의 진화}
	\begin{tabular}{ll}
		\begin{minipage}[t]{.40\textwidth}
			\begin{figure} 
				\includegraphics[trim=10 40 335 450, 
				clip, page=46, width=\textwidth]{\bookfile} 
			\end{figure}
		\end{minipage}
		&
		\begin{minipage}[t]{.55\textwidth}
			\begin{enumerate}\scriptsize 
				\item 1차 원시 대기: 원시 행성을 이루고 있는 대기. 	 \\
				⇒ 수소, 헬륨이 주성분이고 그 외 메테인, 암모니아, 수증기, 이산화 탄소 등
				\item 2차 원시 대기: 원시 행성의 화산폭발로 인해 지구 내부의 가스가 분출됨. \\
				⇒ 수증기, 이산화탄소, 이산화황, 소량 가스로 구성 추정.
			\end{enumerate}
			\questionset {1차 원시대기가 사라진 과정을 설명하시오.}
			\solutionset{수소 헬륨 등의 기체는 초기 지구의 기온이 높아 평균제곱근속도가 매우 커서, 지구 중력이 수소와 헬륨을 잡아둘 정도로 강하지 못하여 중력으로부터 탈출하였고, 나머지 기체도 강력한 태양풍(T-tauri별)에 의해 우주공간으로 날라가서 매우 짧은 기간에 1차 원시 대기를 잃게 됨. \newline}
		
		\end{minipage}		
	\end{tabular}
	\questionset {아웃개싱(outgassing)이란 무엇이며 지구 대기 형성에 어떤 영향을 끼쳤는가?}
	\solutionset{행성 내부에 갖힌 가스가 방출되는 과정으로 오늘날에도 세계 곳곳의 화산에서 일어나고 있다. 지구 초기에는 거대한 열과 유체의 운동에 의해 엄청난 양의 기체가 행성 내부로 부터 공급되었을 것으로 추정된다.}
\end{frame}




\begin{frame}[t]{지구 대기의 진화}
	\begin{tabular}{ll}
		\begin{minipage}[t]{.3\textwidth}
			\begin{figure} 
				\includegraphics[trim=455 360 35 95, 
				clip, page=46, width=\textwidth]{\bookfile} 
			\end{figure}
		\end{minipage}
		&
		\begin{minipage}[t]{.55\textwidth}
			\questionset{2차 원시 대기가 형성된 이후, 산소가 나타나는 과정을 간단히 설명하시오.}
			\solutionset{지구가 식으면서 대기 중의 수증기가 모여 비를 내리고 바다가 형성됨. 바다에서 광합성 박테리아가 산소를 물에 방출하게 되었으며, 초기에는 방출 된 산소가 다른 원자, 분자와 반응하여 쉽게 소비됨(호상 철광층). 유기체의 수가 점차  증가 하면서 산소가 대기중으로 방출되었음.\\
			\newline} 
			
			\questionset{2차 원시 대기에서 상당한 비율을 차지했던 이산화 탄소의 비율이 줄어든 이유를 설명하시오.}
			\solutionset{대기로 방출된 이산화 탄소는 바다의 물과 반응하여 용해되었으며, 여러 단계의 화학과정을 거쳐서 탄산 칼슘이 되었다. 탄산 칼슘은 물에 녹지 않지만, 산호의 껍질과 같은 곳에 이용되고, 석회암을 형성하는데 사용되었다. 이로 인해 현재 대기 중에 존재하는 이산화 탄소의 비율은 2차 원시 대기에 비해 매우 작다.} 				
		\end{minipage}		
	\end{tabular}
\end{frame}





\begin{frame}[t]{변량 기체}
	\begin{tabular}{ll}
		\begin{minipage}[t]{.95\textwidth}
			\begin{enumerate}
				\item 수증기 : 구름과 강수의 근원, 지구복사에너지 흡수, 잠열 수송   
				\item 에어로졸 : 자연 혹은 인간활동으로 인해 생성된 입자가 공기중에 떠다니는 것\\
				ex) 해염, 흙, 연기와 그을음, 꽃가루, 화산폭발 시 먼지 등\\
				⇒ 수증기 응결핵 역할, 태양복사에너지 흡수 및 반사, 광학적 현상 발생원인
				\item  오존 : 대기 중에 차지하는 양은 적으며, 분포가 일정하지 않음. 산소 분자가 자외선을 흡수하면 산소 원자로 쪼개지며, 이 원자가 산소 분자와 충돌하면서 오존이 생성됨. 이때, 촉매가 필요함.\\
				단일 산소 원자를 만들 수 있는 자외선 복사가 충분하며, 충돌이 일어나는데 필요한 기체 분자들이 충분하게 존재하는 약 $10 \sim 50 \rm{~km}$ 고도에 집중(보통 약 $25\rm{~km}$ 고도에서 최대)
			\end{enumerate}
		\end{minipage}
		&		
	\end{tabular}
\end{frame}





\begin{frame}[t]{이산화 탄소 농도(킬링 커브)}
	\begin{tabular}{ll}
		\begin{minipage}[t]{.50\textwidth}
			\centering
			\begin{figure} 
				\includegraphics[trim=260 77 60 415, 
				clip, page=47, width=\textwidth]{\bookfile} 
			\end{figure}
		\end{minipage}
		&
		\begin{minipage}[t]{.40\textwidth}
			\begin{itemize}\scriptsize 
				\item 하와이 마우나 로아(해발 $4169\rm{~m}$) 산 정상에 있는 대기 관측소에서 1958년부터 이산화 탄소의 농도와 지구 기온을 정밀하게 측정
 				\item 찰스 데이비드 킬링(Charles David Keeling, $1928 \sim 2005$)의 이름을 따서 킬링 커브라로 함. 
				\item 세계기상기구(WMO) 지정 
				\item 이산화탄소 세계표준센터인 NOAA가 이곳을 관할
			\end{itemize}
				\questionset{마우나 로아 산 정상에 관측소를 설치한 이유는 무엇인가?}
				\solutionset{태평양 한가운데에 위치하고, 고도가 높아 대도시의 오염물질의 영향을 거의 받지 않아 지구 배경 농도를 측정할 수 있기 때문이다. 
		}
		\end{minipage}
	\end{tabular}
\end{frame}




\begin{frame}[t]{이산화 탄소}
	\begin{tabular}{ll}
		\begin{minipage}[t]{.55\textwidth}
			\centering
			\begin{figure} 
				\includegraphics[trim=260 77 60 420, 
				clip, page=47, width=\textwidth]{\bookfile} 
			\end{figure}
		\end{minipage}
		&
		\begin{minipage}[t]{.35\textwidth}		
			\questionset{$\rm{CO_2}$ 농도가 1년 주기의 패턴을 보이는 이유는 무엇인가?}
			\solutionset{육지가 더 많이 분포하는 북반구가 여름일 때, 식물의 광합성량 증가로 $\rm{CO_2}$ 농도가 낮아지고, 북반구가 겨울일 때 식물의 광합성량이 감소하고, 식물이 부패하며,  난방을 위한 연료 소모 증가로 $\rm{CO_2}$ 농도가 증가함.
			}
		\end{minipage}
	\end{tabular}
\end{frame}





\begin{frame}[t]{에어로졸}
	\begin{tabular}{ll}
		\begin{minipage}[t]{.6\textwidth}
			\begin{figure}
				\includegraphics[trim=58 50 125 390, 
				clip, page=48, width=\textwidth]{\bookfile} 
			\end{figure}
		\end{minipage}
	&
		\begin{minipage}[t]{.35\textwidth}		
			\begin{itemize}
				\item 지구 대기 중을 떠도는 미세한 고체 입자 또는 액체 방울 자연적으로 발생하는 에어로졸인 먼지 폭풍(황사)은 구성 물질의 입자가 큰 편
				\item 인공적인 대기 오염으로 발생하는 에어로졸은 스모그 등이 있음.
			\end{itemize}	
		\end{minipage}
	\end{tabular}
\end{frame}


\begin{frame}[t]{오존의 감소: 전 지구적 문제}
	\begin{tabular}{ll}
		\begin{minipage}[t]{.50\textwidth}
			\begin{figure} 
				\includegraphics[trim=265 400 43 85, 
				clip, page=49, width=\textwidth]{\bookfile} 
			\end{figure}
		\end{minipage}
	&	
		\begin{minipage}[t]{.45\textwidth}
			\begin{itemize}\scriptsize 
				\item CFC는 하층 대기에서 화학적으로 비활성이지만, 오존층에서는 태양복사에너지에 의해 각 구성원소로 분리되며, 이 과정을 통해 염소 이온이 대기 중에 방출됨.
				\item 염소는 촉매 순환 반응에 의해 지속적으로 오존층 파괴 가능
			\end{itemize} \scriptsize
				$$ \begin{array}{l}
				\mathrm{Cl}+\mathrm{O}_{3} \rightarrow \mathrm{ClO}+\mathrm{O}_{2} \\
				\mathrm{ClO}+\mathrm{O} \rightarrow \mathrm{Cl}+\mathrm{O}_{2}
				\end{array}
$$ 

			\questionset {돕슨 단위(Dobson Units)에 대해 설명하시오.}
			\solutionset {돕슨 단위 (Dobson units; DU)로도 표시하는데 1 돕슨은 지구 대기중 오존의 총량을 $0 \rm{{^\circ}C}$, 1 기압의 표준상태에서 두께로 환산했을 때 $0.01\rm{~mm}$에 상당하는 양을 말한다.
			}	
		\end{minipage}
	\end{tabular}
\end{frame}





\section{대기의 연직 구조}


\begin{frame}[t]{고도에 따른 기압의 변화}
	\begin{tabular}{ll}
		\begin{minipage}[t]{.60\textwidth}
			\begin{figure} 
				\includegraphics[trim=205 45 43 400, 
				clip, page=50, width=\textwidth]{\bookfile} 
			\end{figure}
		\end{minipage}
	&
		\begin{minipage}[t]{.350\textwidth}
			\begin{figure} 
				\includegraphics[trim=40 445 365 135, 
				clip, page=51, width=\textwidth]{\bookfile} 
			\end{figure}
			\questionset {제트기가 $10 \rm{~km}$ 상공에서 순항하고 있다. 이곳의 기압은?}
			\solutionset {약 $300 \rm{~hPa}$}
		\end{minipage}		
	\end{tabular}
\end{frame}



\begin{frame}[t]{고도에 따른 기온}
	\begin{tabular}{ll}
		\begin{minipage}[t]{.4\textwidth}
			\begin{figure} 
				\includegraphics[trim=50 45 290 352, 
				clip, page=52, width=\textwidth]{\bookfile} 
			\end{figure}
		\end{minipage}
		&
		\begin{minipage}[t]{.55\textwidth}
			\questionset {대기권을 대류권, 성층권, 중간권, 열권으로 구분하는 기준은 무엇인가?}
			\solutionset {고도에 따른 기온 변화\newline}
			
			\questionset {성층권에서 고도가 상승함에 따라 기온이 증가하는 이유를 설명하시오.}
			\solutionset {오존은 고도 $25 \rm{~km}$ 부근에 가장 많이 분포하지만, 성층권 내에는 오존이 전체적으로 존재하기 때문에 성층권 전역에서 자외선 흡수가 발생한다. \\
			비록 성층권 상층의 오존 밀도가 오존층보다는 작더라도 도달하는 태양복사 에너지가 성층권 상층이 더 많기 때문에 흡수량도 성층권 상층이 더 많음. 그래서 고도가 클수록 기온이 높다.}
		\end{minipage}				
	\end{tabular}
\end{frame}




\begin{frame}[t]{대류권계면의 높이} 
	\begin{tabular}{ll}
		\begin{minipage}[t]{.33\textwidth}
			\centering
			\begin{figure}
				\includegraphics[trim=45 355 370 60, 
				clip, page=53, width=\textwidth]{\bookfile} 
			\end{figure}
		\end{minipage}		
		&
		\begin{minipage}[t]{.620\textwidth}
			\questionset{위도에 따른 대류권계면의 높이는 어떠하며, 그렇게 변하는 이유는 무엇인가?}
			\solutionset{대류권계면의 높이는 저위도에서 높고, 고위도로 갈수록 낮아지는 경향을 보인다. 이 이유는 대류권은 주로 지표에 의해 가열되고 냉각되는데, 지표면의 온도가 높은 저위도에서 더 높은 곳까지 가열하여 영향을 미치기 때문이다. }
		
			\begin{figure} 
				\includegraphics[trim=330 0 0 515, 
				clip, page=51, width=0.4\textwidth]{\bookfile} 
			\end{figure}
			\questionset {높은 산 위에 눈이 보이는 이유를 설명하시오. }
			\solutionset {대류권에서는 평균적으로 약 $6.5 \rm{^\circ}C \rm{~{km}^{-1}}$의 환경감률(environmental lapse rate)을 갖는다. 따라서 높은 고산지대는 영하로 온도가 떨어질 수 있다. }
		\end{minipage}
	\end{tabular}
\end{frame}



\begin{frame}[t]{라디오 존데}
	\begin{tabular}{lll}
		\begin{minipage}[t]{.30\textwidth}
			\begin{figure} 
				\includegraphics[trim=410 330 35 60, 
				clip, page=52, width=\textwidth]{\bookfile} 
			\end{figure}
		\end{minipage}
		&
		\begin{minipage}[t]{.25\textwidth}
			\begin{figure}
				\includegraphics[trim=360 325 90 145, 
				clip, page=53, width=\textwidth]{\bookfile} 
			\end{figure}
		\end{minipage}		
		&
		\begin{minipage}[t]{.35\textwidth}
			\questionset{라디오 존데(radio sonde)가 측정하는 것은?}
			\solutionset{상공의 기압, 기온, 습도, 풍향, 풍속 등을 측정한다. \newline}
		
			\questionset {상층 대기를 관측하기 위해 전세계가 동시에 라디오 존데를 띄워 올린다. 그 시각은 언제인가?}
			\solutionset{세계시로 0시(00UTC)와 12시(12UTC)에 즉 하루에 두번 동시에 띄워 올린다.}
		\end{minipage}
		
	\end{tabular}
\end{frame}



\begin{frame}[t]{전리권}
	\begin{tabular}{ll}
		\begin{minipage}[t]{.50\textwidth}
			\begin{figure} 
				\includegraphics[trim=360 0 0 530, 
				clip, page=54, width=\textwidth]{\bookfile} 
			\end{figure}
		\end{minipage}		
		&
		\begin{minipage}[t]{.450\textwidth}
			\begin{itemize} 
				\item 고도 약 $80 \sim 400 \rm{~km}$에 위치. 전기적으로 대전되어 있는 층.
				\item 질소 분자와 산소 원자들이 태양 에너지를 흡수하여 이온화, 전자는 자유롭게 다니게 됨.
			\end{itemize}
			\begin{block}{극광(aurora)}\scriptsize
				\begin{enumerate}
					\item 태양 플레어 발생시 태양풍 입자들(양자, 전자 등)의 속도와 밀도가 증가.
					\item 이 입자들이 지구에 도달하여 지구 자기장에 붙잡히게 되어 자극 방향으로 흐르는데, 이온들이 원자와 분자에 전류를 흐르게 하여 빛을 발산하게 함. 
				\end{enumerate}
			\end{block}
		\end{minipage}			
	\end{tabular}
\end{frame}



