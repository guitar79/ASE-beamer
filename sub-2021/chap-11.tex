%%%%%%%%%% *** The Title %%%%%%%%%%
\title[]{허리케인\\\small{제11장}}

\begin{frame}[plain] %title page
	\titlepage
\end{frame}


\section{허리케인의 개요}


\begin{frame}[t]{허리케인}
	\begin{tabular}{ll}
		\begin{minipage}[t]{0.45\textwidth}\scriptsize
			\begin{figure}[t]
				\includegraphics[trim=50 290 50 50, clip, page=218, width=\textwidth]{\bookfile}
			\end{figure}
		\end{minipage}	
		&
		\begin{minipage}[t]{0.5\textwidth} \scriptsize	
			열대 및 아열대 해안 주위에 형성.
			열대성 폭풍의 이름은 지역마다 다르다
			: 태풍(태평양 북서부), 사이클론(태평양 남서부 및 인도양), 윌리윌리(오세아니아), 허리케인(북아메리카)
			
			
		\end{minipage}
	\end{tabular}
\end{frame}




\begin{frame}[t]{허리케인 프랜}
	\begin{tabular}{ll}
		\begin{minipage}[t]{0.45\textwidth}\scriptsize
			\begin{figure}[t]
				\includegraphics[trim=50 290 50 50, clip, page=218, width=\textwidth]{\bookfile}
			\end{figure}
		\end{minipage}	
		&
		\begin{minipage}[t]{0.5\textwidth} \scriptsize	
			
			
		\end{minipage}
	\end{tabular}
\end{frame}




\begin{frame}[t]{허리케인}
	\begin{tabular}{ll}
		\begin{minipage}[t]{0.45\textwidth}\scriptsize
			\begin{figure}[t]
				\includegraphics[trim=50 290 50 50, clip, page=218, width=\textwidth]{\bookfile}
			\end{figure}
		\end{minipage}	
		&
		\begin{minipage}[t]{0.5\textwidth} \scriptsize	
			강한 저기압 중심(강한 수렴과 저기압성 회전)을 가지고 있으며, 강한 뇌우 및     120 km/h(약 34 m/s)를 넘는 풍속을 가짐.
			지름이 100~1500k m 정도.(평균 600km)
			강한 기압경도를 가져 풍속이 빠름.
			중심 기압이 950 hPa까지 내려갈 수 있음
			
			
		\end{minipage}
	\end{tabular}
\end{frame}




\begin{frame}[t]{허리케인의 구조}
	\begin{tabular}{ll}
		\begin{minipage}[t]{0.45\textwidth}\scriptsize
			\begin{figure}[t]
				\includegraphics[trim=50 290 50 50, clip, page=218, width=\textwidth]{\bookfile}
			\end{figure}
		\end{minipage}	
		&
		\begin{minipage}[t]{0.5\textwidth} \scriptsize	
			각운동량 보존은 허리케인 중심에서 강한 풍속을 설명
			120 km/h의 풍속은 허리케인 중심에 해당하며, 외곽지역의 바람은 그 보다 풍속이 느리다. 
			허리케인의 중심에서 500 km 떨어진 지점에서 5 km/h의 속력을 가진 공기 덩이가 출발할 경우, 마찰을 무시하면 반경 10 km 근처에서는 250 km/s의 속력을 가짐.
			스케이터의 팔이 축 주위로 원운동하는데, 팔을 안으로 당기면 원운동 반경이 줄어들어 팔은 더 빨라지고 몸이 따라가서 회전속도 증가.
			
			
		\end{minipage}
	\end{tabular}
\end{frame}




\begin{frame}[t]{허리케인의 구조}
	\begin{tabular}{ll}
		\begin{minipage}[t]{0.45\textwidth}\scriptsize
			\begin{figure}[t]
				\includegraphics[trim=50 290 50 50, clip, page=218, width=\textwidth]{\bookfile}
			\end{figure}
		\end{minipage}	
		&
		\begin{minipage}[t]{0.5\textwidth} \scriptsize	
			저기압의 중심에서 수렴하여 상승(적란운 생성) 후, 상층에서 발산
			허리케인의 눈에서 기압은 제일 낮고, 풍속도 느림.
			눈벽 : 폭풍의 중심을 둘러싼 도넛 모양의 벽 -> 최대 풍속, 가장 강한 강우
			강우선 : 점점 약해지는 곡선 이룬 구름층 -> 강우선이 눈벽을 둘러쌈
			태풍의 하층에서는 사이클론(반시계 방향) 순환을 형성하여 중심을 향해 불어 들어 가는 반면, 상층에서는 반사이클론(시계방향) 순환을 형성하여 밖으로 불어 나감.
			
			풍속의 최대 지역은 태풍 중심 부근인 눈벽이며, 중심을 둘러싼 도넛 모양의 벽을 의미한다. 
			이 위치 에서 기압 경도력이 최대가 된다.

			
			
		\end{minipage}
	\end{tabular}
\end{frame}





\section{허리케인의 생성과 소멸}




\begin{frame}[t]{허리케인의 생성}
	\begin{tabular}{ll}
		\begin{minipage}[t]{0.45\textwidth}\scriptsize
			\begin{figure}[t]
				\includegraphics[trim=50 290 50 50, clip, page=218, width=\textwidth]{\bookfile}
			\end{figure}
		\end{minipage}	
		&
		\begin{minipage}[t]{0.5\textwidth} \scriptsize	
			허리케인의 생성 시기 및 위치
			늦여름과 초가을에 자주 발생 (바닷물 온도가 27 ℃ 이상, 열과 수증기 제공)
			위도 30° 이상에서는 생성x (바닷물 온도조건 충족 x 때문)
			적도로부터 5°이내에서는 생성x (전향력이 너무 약하여 회전 운동 발생x)
			
			
		\end{minipage}
	\end{tabular}
\end{frame}




\begin{frame}[t]{제목}
	\begin{tabular}{ll}
		\begin{minipage}[t]{0.45\textwidth}\scriptsize
			\begin{figure}[t]
				\includegraphics[trim=50 290 50 50, clip, page=218, width=\textwidth]{\bookfile}
			\end{figure}
		\end{minipage}	
		&
		\begin{minipage}[t]{0.5\textwidth} \scriptsize	
			열대 요란(불안정한 구름, 뇌우)이 생성
			편동풍 파동에 의해 점진적으로 동쪽에서 서쪽으로 이동
			열대 요란으로부터 형성되는 뇌우에서 잠열이 방출, 요란 내부의 기온이 상승, 밀도가 낮아지게 되고, 지표 기압 감소, 저기압성 순환 발달, 수평기압경도 증가, 지상풍속 증가, 추가 수증기 공급, 폭풍 성장.
			
			
		\end{minipage}
	\end{tabular}
\end{frame}




\begin{frame}[t]{허리케인의 형성}
	\begin{tabular}{ll}
		\begin{minipage}[t]{0.45\textwidth}\scriptsize
			\begin{figure}[t]
				\includegraphics[trim=50 290 50 50, clip, page=218, width=\textwidth]{\bookfile}
			\end{figure}
		\end{minipage}	
		&
		\begin{minipage}[t]{0.5\textwidth} \scriptsize	
			발달 중인 열대성 저기압 상부에서는 대기가 바깥쪽으로 흐르는 데, 만약 이 바람이 없으면 지표면의 압력이 상승되어 추가적 폭풍 발달이 저해됨.
			많은 열대 요란이 발생하지만, 지속적으로 강력한 폭우를 만들지 않으면 90%의 열대요란은 소멸.
			이러한 상황에서 상층에서 고기압이 발달하여 발산이 일어나면 열대성저기압(v < 63 km/h), 열대성폭풍(63 km/h < v < 119km/h), 허리케인(v > 119 km/h)으로 성장.
			낮은 비율의 열대요란이 열대성 저기압으로 진화하지만 큰 규모의 열대성 저기압은 열대성 폭풍이 되고, 훨씬 큰 규모의 열대성 폭풍은 허리케인으로 강화.
			
			
		\end{minipage}
	\end{tabular}
\end{frame}




\begin{frame}[t]{허리케인의 소멸}
	\begin{tabular}{ll}
		\begin{minipage}[t]{0.45\textwidth}\scriptsize
			잠열이 차단되면 소멸됨
			허리케인이 차가운 물이나 육지로 이동하는 경우
			상층의 큰 규모의 흐름이 허리케인의 성장에 유리하지 않을 때
			
			Q) 허리케인은 어떻게 소멸하는가?
			허리케인의 에너지원인 수증기 공급이 줄어들면 세력이 약화된다.
			1) 육지로 이동하는 경우
			2) 찬 해수 지역으로 이동하는 경우
			3) 해수의 온도는 높으나 온도가 높은 층의 깊이가 얕은 경우
			
		\end{minipage}	
		&
		\begin{minipage}[t]{0.5\textwidth} \scriptsize	
			Q) 열대성 폭풍과 열대성 저압부 중 풍속이 강한 것은 무엇인가?
			열대성 저압부은 풍속이 63 km/h보다 느린 경우를, 열대성 폭풍은 63 km/h보다 빠르고 119 km/h보다는 느린 경우를 말한다. 우리나라에서는 열대성 폭풍 이상을 보통 태풍이라고 부른다. 미국에서는 119 km/h 이상의 풍속을 가진 경우 허리케인이라고 부른다.
			
			Q) 적도 근처에서 형성되는 열대성 폭풍은 고위도의 저기압처럼 회전운동을 하지 않는 이유는 무엇인지 설명하시오.
			적도 근처는 코리올리 힘이 매우 작기 때문에 고위도의 저기압과 같은 회전운동을 하지 않는다. 대신 편동풍 파동에 의해서 저기압성 회전이 발달할 수 있다. 
			
			
			
		\end{minipage}
	\end{tabular}
\end{frame}




\begin{frame}[t]{제목}
	\begin{tabular}{ll}
		\begin{minipage}[t]{0.45\textwidth}\scriptsize
			Q) 열대 요란의 강화를 방해하는 두 가지 요소를 쓰시오.
			1)무역풍 역전
			아열대고기압의 영향을 받은 지역에서 나타나는 침강으로 인하여 형성된 강한 역전은 대기의 상승력을 감소시키고 뇌우의 발생을 억제
			2) 강한 상층 바람
			상층의 강한 바람은 구름 상부에서 발산된 잠열을 소산되게 하여 열대 요란의 지속적인 성장과 발달을 방해
			
			Q) 허리케인의 강도는 육지로 이동할 때 왜 빠르게 약화되는지 설명하시오.
			허리케인이 육지로 이동하게 되면 따뜻함과 수증기의 공급원이 차단된다. 일반적인 경우 육지가 바다보다 더 빨리 냉각이 일어나게 되어 하층 공기가 가열되기 보다 냉각된다. 뿐만 아니라 마찰이 증가하여 표층 풍속이 급속히 감소하게 되고, 저기압 중심으로 바람이 보다 직접적으로 들어가게 되어 기압차가 감소하게 되어 약화된다.
			
			
			
			
		\end{minipage}	
		&
		\begin{minipage}[t]{0.5\textwidth} \scriptsize	
			
			
		\end{minipage}
	\end{tabular}
\end{frame}




\begin{frame}[t]{허리케인의 명칭}
	\begin{tabular}{ll}
		\begin{minipage}[t]{0.45\textwidth}\scriptsize
			\begin{figure}[t]
				\includegraphics[trim=50 290 50 50, clip, page=218, width=\textwidth]{\bookfile}
			\end{figure}
		\end{minipage}	
		&
		\begin{minipage}[t]{0.5\textwidth} \scriptsize	
			Q) 과거 열대성 저기압의 이름은 어떻게 명명하였는가?
			싫어하는 정치인의 이름, 여성의 이름 등
			
			
		\end{minipage}
	\end{tabular}
\end{frame}




\begin{frame}[t]{제목}
	\begin{tabular}{ll}
		\begin{minipage}[t]{0.45\textwidth}\scriptsize
			\begin{figure}[t]
				\includegraphics[trim=50 290 50 50, clip, page=218, width=\textwidth]{\bookfile}
			\end{figure}
		\end{minipage}	
		&
		\begin{minipage}[t]{0.5\textwidth} \scriptsize	
			Q) 태풍의 이름은 모두 몇개인가?
			14개 나라 각 10개씩 140개를 5개 조로 나눈다.
			Q) 태풍의 이름은 어떻게 퇴출되는가??
			심각한 피해를 입힌 태풍의 경우 회원국이 요청을하면 영구 제외되고 새로운 이름으로 교체된다.
			
			
		\end{minipage}
	\end{tabular}
\end{frame}






\section{허리케인의 피해}



\begin{frame}[t]{제목}
	\begin{tabular}{ll}
		\begin{minipage}[t]{0.45\textwidth}\scriptsize
			\begin{figure}[t]
				\includegraphics[trim=50 290 50 50, clip, page=218, width=\textwidth]{\bookfile}
			\end{figure}
		\end{minipage}	
		&
		\begin{minipage}[t]{0.5\textwidth} \scriptsize	
			
			
		\end{minipage}
	\end{tabular}
\end{frame}


\begin{frame}[t]{제목}
	\begin{tabular}{ll}
		\begin{minipage}[t]{0.45\textwidth}\scriptsize
			\begin{figure}[t]
				\includegraphics[trim=50 290 50 50, clip, page=218, width=\textwidth]{\bookfile}
			\end{figure}
		\end{minipage}	
		&
		\begin{minipage}[t]{0.5\textwidth} \scriptsize	
			
			
		\end{minipage}
	\end{tabular}
\end{frame}




\begin{frame}[t]{제목}
	\begin{tabular}{ll}
		\begin{minipage}[t]{0.45\textwidth}\scriptsize
			\begin{figure}[t]
				\includegraphics[trim=50 290 50 50, clip, page=218, width=\textwidth]{\bookfile}
			\end{figure}
		\end{minipage}	
		&
		\begin{minipage}[t]{0.5\textwidth} \scriptsize	
			
			
		\end{minipage}
	\end{tabular}
\end{frame}




\begin{frame}[t]{제목}
	\begin{tabular}{ll}
		\begin{minipage}[t]{0.45\textwidth}\scriptsize
			\begin{figure}[t]
				\includegraphics[trim=50 290 50 50, clip, page=218, width=\textwidth]{\bookfile}
			\end{figure}
		\end{minipage}	
		&
		\begin{minipage}[t]{0.5\textwidth} \scriptsize	
			
			
		\end{minipage}
	\end{tabular}
\end{frame}




\begin{frame}[t]{제목}
	\begin{tabular}{ll}
		\begin{minipage}[t]{0.45\textwidth}\scriptsize
			\begin{figure}[t]
				\includegraphics[trim=50 290 50 50, clip, page=218, width=\textwidth]{\bookfile}
			\end{figure}
		\end{minipage}	
		&
		\begin{minipage}[t]{0.5\textwidth} \scriptsize	
			
			
		\end{minipage}
	\end{tabular}
\end{frame}




\begin{frame}[t]{제목}
	\begin{tabular}{ll}
		\begin{minipage}[t]{0.45\textwidth}\scriptsize
			\begin{figure}[t]
				\includegraphics[trim=50 290 50 50, clip, page=218, width=\textwidth]{\bookfile}
			\end{figure}
		\end{minipage}	
		&
		\begin{minipage}[t]{0.5\textwidth} \scriptsize	
			
			
		\end{minipage}
	\end{tabular}
\end{frame}




\begin{frame}[t]{제목}
	\begin{tabular}{ll}
		\begin{minipage}[t]{0.45\textwidth}\scriptsize
			\begin{figure}[t]
				\includegraphics[trim=50 290 50 50, clip, page=218, width=\textwidth]{\bookfile}
			\end{figure}
		\end{minipage}	
		&
		\begin{minipage}[t]{0.5\textwidth} \scriptsize	
			
			
		\end{minipage}
	\end{tabular}
\end{frame}




\begin{frame}[t]{제목}
	\begin{tabular}{ll}
		\begin{minipage}[t]{0.45\textwidth}\scriptsize
			\begin{figure}[t]
				\includegraphics[trim=50 290 50 50, clip, page=218, width=\textwidth]{\bookfile}
			\end{figure}
		\end{minipage}	
		&
		\begin{minipage}[t]{0.5\textwidth} \scriptsize	
			
			
		\end{minipage}
	\end{tabular}
\end{frame}




\begin{frame}[t]{제목}
	\begin{tabular}{ll}
		\begin{minipage}[t]{0.45\textwidth}\scriptsize
			\begin{figure}[t]
				\includegraphics[trim=50 290 50 50, clip, page=218, width=\textwidth]{\bookfile}
			\end{figure}
		\end{minipage}	
		&
		\begin{minipage}[t]{0.5\textwidth} \scriptsize	
			
			
		\end{minipage}
	\end{tabular}
\end{frame}



\section{허리케인의 강도 추정}




\begin{frame}[t]{제목}
	\begin{tabular}{ll}
		\begin{minipage}[t]{0.45\textwidth}\scriptsize
			\begin{figure}[t]
				\includegraphics[trim=50 290 50 50, clip, page=218, width=\textwidth]{\bookfile}
			\end{figure}
		\end{minipage}	
		&
		\begin{minipage}[t]{0.5\textwidth} \scriptsize	
			
			
		\end{minipage}
	\end{tabular}
\end{frame}




\begin{frame}[t]{제목}
	\begin{tabular}{ll}
		\begin{minipage}[t]{0.45\textwidth}\scriptsize
			\begin{figure}[t]
				\includegraphics[trim=50 290 50 50, clip, page=218, width=\textwidth]{\bookfile}
			\end{figure}
		\end{minipage}	
		&
		\begin{minipage}[t]{0.5\textwidth} \scriptsize	
			
			
		\end{minipage}
	\end{tabular}
\end{frame}




\begin{frame}[t]{제목}
	\begin{tabular}{ll}
		\begin{minipage}[t]{0.45\textwidth}\scriptsize
			\begin{figure}[t]
				\includegraphics[trim=50 290 50 50, clip, page=218, width=\textwidth]{\bookfile}
			\end{figure}
		\end{minipage}	
		&
		\begin{minipage}[t]{0.5\textwidth} \scriptsize	
			
			
		\end{minipage}
	\end{tabular}
\end{frame}




\begin{frame}[t]{제목}
	\begin{tabular}{ll}
		\begin{minipage}[t]{0.45\textwidth}\scriptsize
			\begin{figure}[t]
				\includegraphics[trim=50 290 50 50, clip, page=218, width=\textwidth]{\bookfile}
			\end{figure}
		\end{minipage}	
		&
		\begin{minipage}[t]{0.5\textwidth} \scriptsize	
			
			
		\end{minipage}
	\end{tabular}
\end{frame}




\begin{frame}[t]{제목}
	\begin{tabular}{ll}
		\begin{minipage}[t]{0.45\textwidth}\scriptsize
			\begin{figure}[t]
				\includegraphics[trim=50 290 50 50, clip, page=218, width=\textwidth]{\bookfile}
			\end{figure}
		\end{minipage}	
		&
		\begin{minipage}[t]{0.5\textwidth} \scriptsize	
			
			
		\end{minipage}
	\end{tabular}
\end{frame}




\begin{frame}[t]{제목}
	\begin{tabular}{ll}
		\begin{minipage}[t]{0.45\textwidth}\scriptsize
			\begin{figure}[t]
				\includegraphics[trim=50 290 50 50, clip, page=218, width=\textwidth]{\bookfile}
			\end{figure}
		\end{minipage}	
		&
		\begin{minipage}[t]{0.5\textwidth} \scriptsize	
			
			
		\end{minipage}
	\end{tabular}
\end{frame}




\begin{frame}[t]{제목}
	\begin{tabular}{ll}
		\begin{minipage}[t]{0.45\textwidth}\scriptsize
			\begin{figure}[t]
				\includegraphics[trim=50 290 50 50, clip, page=218, width=\textwidth]{\bookfile}
			\end{figure}
		\end{minipage}	
		&
		\begin{minipage}[t]{0.5\textwidth} \scriptsize	
			
			
		\end{minipage}
	\end{tabular}
\end{frame}




\begin{frame}[t]{제목}
	\begin{tabular}{ll}
		\begin{minipage}[t]{0.45\textwidth}\scriptsize
			\begin{figure}[t]
				\includegraphics[trim=50 290 50 50, clip, page=218, width=\textwidth]{\bookfile}
			\end{figure}
		\end{minipage}	
		&
		\begin{minipage}[t]{0.5\textwidth} \scriptsize	
			
			
		\end{minipage}
	\end{tabular}
\end{frame}



\section{허리케인 탐지와 진로 분석}



\begin{frame}[t]{제목}
	\begin{tabular}{ll}
		\begin{minipage}[t]{0.45\textwidth}\scriptsize
			\begin{figure}[t]
				\includegraphics[trim=50 290 50 50, clip, page=218, width=\textwidth]{\bookfile}
			\end{figure}
		\end{minipage}	
		&
		\begin{minipage}[t]{0.5\textwidth} \scriptsize	
			
			
		\end{minipage}
	\end{tabular}
\end{frame}




\begin{frame}[t]{제목}
	\begin{tabular}{ll}
		\begin{minipage}[t]{0.45\textwidth}\scriptsize
			\begin{figure}[t]
				\includegraphics[trim=50 290 50 50, clip, page=218, width=\textwidth]{\bookfile}
			\end{figure}
		\end{minipage}	
		&
		\begin{minipage}[t]{0.5\textwidth} \scriptsize	
			
			
		\end{minipage}
	\end{tabular}
\end{frame}




\begin{frame}[t]{제목}
	\begin{tabular}{ll}
		\begin{minipage}[t]{0.45\textwidth}\scriptsize
			\begin{figure}[t]
				\includegraphics[trim=50 290 50 50, clip, page=218, width=\textwidth]{\bookfile}
			\end{figure}
		\end{minipage}	
		&
		\begin{minipage}[t]{0.5\textwidth} \scriptsize	
			
			
		\end{minipage}
	\end{tabular}
\end{frame}




\begin{frame}[t]{제목}
	\begin{tabular}{ll}
		\begin{minipage}[t]{0.45\textwidth}\scriptsize
			\begin{figure}[t]
				\includegraphics[trim=50 290 50 50, clip, page=218, width=\textwidth]{\bookfile}
			\end{figure}
		\end{minipage}	
		&
		\begin{minipage}[t]{0.5\textwidth} \scriptsize	
			
			
		\end{minipage}
	\end{tabular}
\end{frame}




\begin{frame}[t]{제목}
	\begin{tabular}{ll}
		\begin{minipage}[t]{0.45\textwidth}\scriptsize
			\begin{figure}[t]
				\includegraphics[trim=50 290 50 50, clip, page=218, width=\textwidth]{\bookfile}
			\end{figure}
		\end{minipage}	
		&
		\begin{minipage}[t]{0.5\textwidth} \scriptsize	
			
			
		\end{minipage}
	\end{tabular}
\end{frame}




\begin{frame}[t]{제목}
	\begin{tabular}{ll}
		\begin{minipage}[t]{0.45\textwidth}\scriptsize
			\begin{figure}[t]
				\includegraphics[trim=50 290 50 50, clip, page=218, width=\textwidth]{\bookfile}
			\end{figure}
		\end{minipage}	
		&
		\begin{minipage}[t]{0.5\textwidth} \scriptsize	
			
			
		\end{minipage}
	\end{tabular}
\end{frame}




\begin{frame}[t]{제목}
	\begin{tabular}{ll}
		\begin{minipage}[t]{0.45\textwidth}\scriptsize
			\begin{figure}[t]
				\includegraphics[trim=50 290 50 50, clip, page=218, width=\textwidth]{\bookfile}
			\end{figure}
		\end{minipage}	
		&
		\begin{minipage}[t]{0.5\textwidth} \scriptsize	
			
			
		\end{minipage}
	\end{tabular}
\end{frame}




\begin{frame}[t]{제목}
	\begin{tabular}{ll}
		\begin{minipage}[t]{0.45\textwidth}\scriptsize
			\begin{figure}[t]
				\includegraphics[trim=50 290 50 50, clip, page=218, width=\textwidth]{\bookfile}
			\end{figure}
		\end{minipage}	
		&
		\begin{minipage}[t]{0.5\textwidth} \scriptsize	
			
			
		\end{minipage}
	\end{tabular}
\end{frame}



